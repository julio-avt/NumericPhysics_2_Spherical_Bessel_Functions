\documentclass[9pt]{article}
	\usepackage[a4paper, total={7in, 8in}]{geometry}
    \usepackage[breakable]{tcolorbox}
    \usepackage{parskip} % Stop auto-indenting (to mimic markdown behaviour)
    
    \usepackage{iftex}
    \ifPDFTeX
    	\usepackage[T1]{fontenc}
    	\usepackage{mathpazo}
    \else
    	\usepackage{fontspec}
    \fi

    % Basic figure setup, for now with no caption control since it's done
    % automatically by Pandoc (which extracts ![](path) syntax from Markdown).
    \usepackage{graphicx}
    % Maintain compatibility with old templates. Remove in nbconvert 6.0
    \let\Oldincludegraphics\includegraphics
    % Ensure that by default, figures have no caption (until we provide a
    % proper Figure object with a Caption API and a way to capture that
    % in the conversion process - todo).
    \usepackage{caption}
    \DeclareCaptionFormat{nocaption}{}
    \captionsetup{format=nocaption,aboveskip=0pt,belowskip=0pt}

    \usepackage{float}
    \floatplacement{figure}{H} % forces figures to be placed at the correct location
    \usepackage{xcolor} % Allow colors to be defined
    \usepackage{enumerate} % Needed for markdown enumerations to work
    \usepackage{geometry} % Used to adjust the document margins
    \usepackage{amsmath} % Equations
    \usepackage{amssymb} % Equations
    \usepackage{textcomp} % defines textquotesingle
    % Hack from http://tex.stackexchange.com/a/47451/13684:
    \AtBeginDocument{%
        \def\PYZsq{\textquotesingle}% Upright quotes in Pygmentized code
    }
    \usepackage{upquote} % Upright quotes for verbatim code
    \usepackage{eurosym} % defines \euro
    \usepackage[mathletters]{ucs} % Extended unicode (utf-8) support
    \usepackage{fancyvrb} % verbatim replacement that allows latex
    \usepackage{grffile} % extends the file name processing of package graphics 
                         % to support a larger range
    \makeatletter % fix for old versions of grffile with XeLaTeX
    \@ifpackagelater{grffile}{2019/11/01}
    {
      % Do nothing on new versions
    }
    {
      \def\Gread@@xetex#1{%
        \IfFileExists{"\Gin@base".bb}%
        {\Gread@eps{\Gin@base.bb}}%
        {\Gread@@xetex@aux#1}%
      }
    }
    \makeatother
    \usepackage[Export]{adjustbox} % Used to constrain images to a maximum size
    \adjustboxset{max size={0.9\linewidth}{0.9\paperheight}}

    % The hyperref package gives us a pdf with properly built
    % internal navigation ('pdf bookmarks' for the table of contents,
    % internal cross-reference links, web links for URLs, etc.)
    \usepackage{hyperref}
    % The default LaTeX title has an obnoxious amount of whitespace. By default,
    % titling removes some of it. It also provides customization options.
    \usepackage{titling}
    \usepackage{longtable} % longtable support required by pandoc >1.10
    \usepackage{booktabs}  % table support for pandoc > 1.12.2
    \usepackage[inline]{enumitem} % IRkernel/repr support (it uses the enumerate* environment)
    \usepackage[normalem]{ulem} % ulem is needed to support strikethroughs (\sout)
                                % normalem makes italics be italics, not underlines
    \usepackage{mathrsfs}
    

    
    % Colors for the hyperref package
    \definecolor{urlcolor}{rgb}{0,.145,.698}
    \definecolor{linkcolor}{rgb}{.71,0.21,0.01}
    \definecolor{citecolor}{rgb}{.12,.54,.11}

    % ANSI colors
    \definecolor{ansi-black}{HTML}{3E424D}
    \definecolor{ansi-black-intense}{HTML}{282C36}
    \definecolor{ansi-red}{HTML}{E75C58}
    \definecolor{ansi-red-intense}{HTML}{B22B31}
    \definecolor{ansi-green}{HTML}{00A250}
    \definecolor{ansi-green-intense}{HTML}{007427}
    \definecolor{ansi-yellow}{HTML}{DDB62B}
    \definecolor{ansi-yellow-intense}{HTML}{B27D12}
    \definecolor{ansi-blue}{HTML}{208FFB}
    \definecolor{ansi-blue-intense}{HTML}{0065CA}
    \definecolor{ansi-magenta}{HTML}{D160C4}
    \definecolor{ansi-magenta-intense}{HTML}{A03196}
    \definecolor{ansi-cyan}{HTML}{60C6C8}
    \definecolor{ansi-cyan-intense}{HTML}{258F8F}
    \definecolor{ansi-white}{HTML}{C5C1B4}
    \definecolor{ansi-white-intense}{HTML}{A1A6B2}
    \definecolor{ansi-default-inverse-fg}{HTML}{FFFFFF}
    \definecolor{ansi-default-inverse-bg}{HTML}{000000}

    % common color for the border for error outputs.
    \definecolor{outerrorbackground}{HTML}{FFDFDF}

    % commands and environments needed by pandoc snippets
    % extracted from the output of `pandoc -s`
    \providecommand{\tightlist}{%
      \setlength{\itemsep}{0pt}\setlength{\parskip}{0pt}}
    \DefineVerbatimEnvironment{Highlighting}{Verbatim}{commandchars=\\\{\}}
    % Add ',fontsize=\small' for more characters per line
    \newenvironment{Shaded}{}{}
    \newcommand{\KeywordTok}[1]{\textcolor[rgb]{0.00,0.44,0.13}{\textbf{{#1}}}}
    \newcommand{\DataTypeTok}[1]{\textcolor[rgb]{0.56,0.13,0.00}{{#1}}}
    \newcommand{\DecValTok}[1]{\textcolor[rgb]{0.25,0.63,0.44}{{#1}}}
    \newcommand{\BaseNTok}[1]{\textcolor[rgb]{0.25,0.63,0.44}{{#1}}}
    \newcommand{\FloatTok}[1]{\textcolor[rgb]{0.25,0.63,0.44}{{#1}}}
    \newcommand{\CharTok}[1]{\textcolor[rgb]{0.25,0.44,0.63}{{#1}}}
    \newcommand{\StringTok}[1]{\textcolor[rgb]{0.25,0.44,0.63}{{#1}}}
    \newcommand{\CommentTok}[1]{\textcolor[rgb]{0.38,0.63,0.69}{\textit{{#1}}}}
    \newcommand{\OtherTok}[1]{\textcolor[rgb]{0.00,0.44,0.13}{{#1}}}
    \newcommand{\AlertTok}[1]{\textcolor[rgb]{1.00,0.00,0.00}{\textbf{{#1}}}}
    \newcommand{\FunctionTok}[1]{\textcolor[rgb]{0.02,0.16,0.49}{{#1}}}
    \newcommand{\RegionMarkerTok}[1]{{#1}}
    \newcommand{\ErrorTok}[1]{\textcolor[rgb]{1.00,0.00,0.00}{\textbf{{#1}}}}
    \newcommand{\NormalTok}[1]{{#1}}
    
    % Additional commands for more recent versions of Pandoc
    \newcommand{\ConstantTok}[1]{\textcolor[rgb]{0.53,0.00,0.00}{{#1}}}
    \newcommand{\SpecialCharTok}[1]{\textcolor[rgb]{0.25,0.44,0.63}{{#1}}}
    \newcommand{\VerbatimStringTok}[1]{\textcolor[rgb]{0.25,0.44,0.63}{{#1}}}
    \newcommand{\SpecialStringTok}[1]{\textcolor[rgb]{0.73,0.40,0.53}{{#1}}}
    \newcommand{\ImportTok}[1]{{#1}}
    \newcommand{\DocumentationTok}[1]{\textcolor[rgb]{0.73,0.13,0.13}{\textit{{#1}}}}
    \newcommand{\AnnotationTok}[1]{\textcolor[rgb]{0.38,0.63,0.69}{\textbf{\textit{{#1}}}}}
    \newcommand{\CommentVarTok}[1]{\textcolor[rgb]{0.38,0.63,0.69}{\textbf{\textit{{#1}}}}}
    \newcommand{\VariableTok}[1]{\textcolor[rgb]{0.10,0.09,0.49}{{#1}}}
    \newcommand{\ControlFlowTok}[1]{\textcolor[rgb]{0.00,0.44,0.13}{\textbf{{#1}}}}
    \newcommand{\OperatorTok}[1]{\textcolor[rgb]{0.40,0.40,0.40}{{#1}}}
    \newcommand{\BuiltInTok}[1]{{#1}}
    \newcommand{\ExtensionTok}[1]{{#1}}
    \newcommand{\PreprocessorTok}[1]{\textcolor[rgb]{0.74,0.48,0.00}{{#1}}}
    \newcommand{\AttributeTok}[1]{\textcolor[rgb]{0.49,0.56,0.16}{{#1}}}
    \newcommand{\InformationTok}[1]{\textcolor[rgb]{0.38,0.63,0.69}{\textbf{\textit{{#1}}}}}
    \newcommand{\WarningTok}[1]{\textcolor[rgb]{0.38,0.63,0.69}{\textbf{\textit{{#1}}}}}
    
    
    % Define a nice break command that doesn't care if a line doesn't already
    % exist.
    \def\br{\hspace*{\fill} \\* }
    % Math Jax compatibility definitions
    \def\gt{>}
    \def\lt{<}
    \let\Oldtex\TeX
    \let\Oldlatex\LaTeX
    \renewcommand{\TeX}{\textrm{\Oldtex}}
    \renewcommand{\LaTeX}{\textrm{\Oldlatex}}
    % Document parameters
    % Document title
    \title{Tarea2\_Explicacion}
    
    
    
    
    
% Pygments definitions
\makeatletter
\def\PY@reset{\let\PY@it=\relax \let\PY@bf=\relax%
    \let\PY@ul=\relax \let\PY@tc=\relax%
    \let\PY@bc=\relax \let\PY@ff=\relax}
\def\PY@tok#1{\csname PY@tok@#1\endcsname}
\def\PY@toks#1+{\ifx\relax#1\empty\else%
    \PY@tok{#1}\expandafter\PY@toks\fi}
\def\PY@do#1{\PY@bc{\PY@tc{\PY@ul{%
    \PY@it{\PY@bf{\PY@ff{#1}}}}}}}
\def\PY#1#2{\PY@reset\PY@toks#1+\relax+\PY@do{#2}}

\@namedef{PY@tok@w}{\def\PY@tc##1{\textcolor[rgb]{0.73,0.73,0.73}{##1}}}
\@namedef{PY@tok@c}{\let\PY@it=\textit\def\PY@tc##1{\textcolor[rgb]{0.25,0.50,0.50}{##1}}}
\@namedef{PY@tok@cp}{\def\PY@tc##1{\textcolor[rgb]{0.74,0.48,0.00}{##1}}}
\@namedef{PY@tok@k}{\let\PY@bf=\textbf\def\PY@tc##1{\textcolor[rgb]{0.00,0.50,0.00}{##1}}}
\@namedef{PY@tok@kp}{\def\PY@tc##1{\textcolor[rgb]{0.00,0.50,0.00}{##1}}}
\@namedef{PY@tok@kt}{\def\PY@tc##1{\textcolor[rgb]{0.69,0.00,0.25}{##1}}}
\@namedef{PY@tok@o}{\def\PY@tc##1{\textcolor[rgb]{0.40,0.40,0.40}{##1}}}
\@namedef{PY@tok@ow}{\let\PY@bf=\textbf\def\PY@tc##1{\textcolor[rgb]{0.67,0.13,1.00}{##1}}}
\@namedef{PY@tok@nb}{\def\PY@tc##1{\textcolor[rgb]{0.00,0.50,0.00}{##1}}}
\@namedef{PY@tok@nf}{\def\PY@tc##1{\textcolor[rgb]{0.00,0.00,1.00}{##1}}}
\@namedef{PY@tok@nc}{\let\PY@bf=\textbf\def\PY@tc##1{\textcolor[rgb]{0.00,0.00,1.00}{##1}}}
\@namedef{PY@tok@nn}{\let\PY@bf=\textbf\def\PY@tc##1{\textcolor[rgb]{0.00,0.00,1.00}{##1}}}
\@namedef{PY@tok@ne}{\let\PY@bf=\textbf\def\PY@tc##1{\textcolor[rgb]{0.82,0.25,0.23}{##1}}}
\@namedef{PY@tok@nv}{\def\PY@tc##1{\textcolor[rgb]{0.10,0.09,0.49}{##1}}}
\@namedef{PY@tok@no}{\def\PY@tc##1{\textcolor[rgb]{0.53,0.00,0.00}{##1}}}
\@namedef{PY@tok@nl}{\def\PY@tc##1{\textcolor[rgb]{0.63,0.63,0.00}{##1}}}
\@namedef{PY@tok@ni}{\let\PY@bf=\textbf\def\PY@tc##1{\textcolor[rgb]{0.60,0.60,0.60}{##1}}}
\@namedef{PY@tok@na}{\def\PY@tc##1{\textcolor[rgb]{0.49,0.56,0.16}{##1}}}
\@namedef{PY@tok@nt}{\let\PY@bf=\textbf\def\PY@tc##1{\textcolor[rgb]{0.00,0.50,0.00}{##1}}}
\@namedef{PY@tok@nd}{\def\PY@tc##1{\textcolor[rgb]{0.67,0.13,1.00}{##1}}}
\@namedef{PY@tok@s}{\def\PY@tc##1{\textcolor[rgb]{0.73,0.13,0.13}{##1}}}
\@namedef{PY@tok@sd}{\let\PY@it=\textit\def\PY@tc##1{\textcolor[rgb]{0.73,0.13,0.13}{##1}}}
\@namedef{PY@tok@si}{\let\PY@bf=\textbf\def\PY@tc##1{\textcolor[rgb]{0.73,0.40,0.53}{##1}}}
\@namedef{PY@tok@se}{\let\PY@bf=\textbf\def\PY@tc##1{\textcolor[rgb]{0.73,0.40,0.13}{##1}}}
\@namedef{PY@tok@sr}{\def\PY@tc##1{\textcolor[rgb]{0.73,0.40,0.53}{##1}}}
\@namedef{PY@tok@ss}{\def\PY@tc##1{\textcolor[rgb]{0.10,0.09,0.49}{##1}}}
\@namedef{PY@tok@sx}{\def\PY@tc##1{\textcolor[rgb]{0.00,0.50,0.00}{##1}}}
\@namedef{PY@tok@m}{\def\PY@tc##1{\textcolor[rgb]{0.40,0.40,0.40}{##1}}}
\@namedef{PY@tok@gh}{\let\PY@bf=\textbf\def\PY@tc##1{\textcolor[rgb]{0.00,0.00,0.50}{##1}}}
\@namedef{PY@tok@gu}{\let\PY@bf=\textbf\def\PY@tc##1{\textcolor[rgb]{0.50,0.00,0.50}{##1}}}
\@namedef{PY@tok@gd}{\def\PY@tc##1{\textcolor[rgb]{0.63,0.00,0.00}{##1}}}
\@namedef{PY@tok@gi}{\def\PY@tc##1{\textcolor[rgb]{0.00,0.63,0.00}{##1}}}
\@namedef{PY@tok@gr}{\def\PY@tc##1{\textcolor[rgb]{1.00,0.00,0.00}{##1}}}
\@namedef{PY@tok@ge}{\let\PY@it=\textit}
\@namedef{PY@tok@gs}{\let\PY@bf=\textbf}
\@namedef{PY@tok@gp}{\let\PY@bf=\textbf\def\PY@tc##1{\textcolor[rgb]{0.00,0.00,0.50}{##1}}}
\@namedef{PY@tok@go}{\def\PY@tc##1{\textcolor[rgb]{0.53,0.53,0.53}{##1}}}
\@namedef{PY@tok@gt}{\def\PY@tc##1{\textcolor[rgb]{0.00,0.27,0.87}{##1}}}
\@namedef{PY@tok@err}{\def\PY@bc##1{{\setlength{\fboxsep}{\string -\fboxrule}\fcolorbox[rgb]{1.00,0.00,0.00}{1,1,1}{\strut ##1}}}}
\@namedef{PY@tok@kc}{\let\PY@bf=\textbf\def\PY@tc##1{\textcolor[rgb]{0.00,0.50,0.00}{##1}}}
\@namedef{PY@tok@kd}{\let\PY@bf=\textbf\def\PY@tc##1{\textcolor[rgb]{0.00,0.50,0.00}{##1}}}
\@namedef{PY@tok@kn}{\let\PY@bf=\textbf\def\PY@tc##1{\textcolor[rgb]{0.00,0.50,0.00}{##1}}}
\@namedef{PY@tok@kr}{\let\PY@bf=\textbf\def\PY@tc##1{\textcolor[rgb]{0.00,0.50,0.00}{##1}}}
\@namedef{PY@tok@bp}{\def\PY@tc##1{\textcolor[rgb]{0.00,0.50,0.00}{##1}}}
\@namedef{PY@tok@fm}{\def\PY@tc##1{\textcolor[rgb]{0.00,0.00,1.00}{##1}}}
\@namedef{PY@tok@vc}{\def\PY@tc##1{\textcolor[rgb]{0.10,0.09,0.49}{##1}}}
\@namedef{PY@tok@vg}{\def\PY@tc##1{\textcolor[rgb]{0.10,0.09,0.49}{##1}}}
\@namedef{PY@tok@vi}{\def\PY@tc##1{\textcolor[rgb]{0.10,0.09,0.49}{##1}}}
\@namedef{PY@tok@vm}{\def\PY@tc##1{\textcolor[rgb]{0.10,0.09,0.49}{##1}}}
\@namedef{PY@tok@sa}{\def\PY@tc##1{\textcolor[rgb]{0.73,0.13,0.13}{##1}}}
\@namedef{PY@tok@sb}{\def\PY@tc##1{\textcolor[rgb]{0.73,0.13,0.13}{##1}}}
\@namedef{PY@tok@sc}{\def\PY@tc##1{\textcolor[rgb]{0.73,0.13,0.13}{##1}}}
\@namedef{PY@tok@dl}{\def\PY@tc##1{\textcolor[rgb]{0.73,0.13,0.13}{##1}}}
\@namedef{PY@tok@s2}{\def\PY@tc##1{\textcolor[rgb]{0.73,0.13,0.13}{##1}}}
\@namedef{PY@tok@sh}{\def\PY@tc##1{\textcolor[rgb]{0.73,0.13,0.13}{##1}}}
\@namedef{PY@tok@s1}{\def\PY@tc##1{\textcolor[rgb]{0.73,0.13,0.13}{##1}}}
\@namedef{PY@tok@mb}{\def\PY@tc##1{\textcolor[rgb]{0.40,0.40,0.40}{##1}}}
\@namedef{PY@tok@mf}{\def\PY@tc##1{\textcolor[rgb]{0.40,0.40,0.40}{##1}}}
\@namedef{PY@tok@mh}{\def\PY@tc##1{\textcolor[rgb]{0.40,0.40,0.40}{##1}}}
\@namedef{PY@tok@mi}{\def\PY@tc##1{\textcolor[rgb]{0.40,0.40,0.40}{##1}}}
\@namedef{PY@tok@il}{\def\PY@tc##1{\textcolor[rgb]{0.40,0.40,0.40}{##1}}}
\@namedef{PY@tok@mo}{\def\PY@tc##1{\textcolor[rgb]{0.40,0.40,0.40}{##1}}}
\@namedef{PY@tok@ch}{\let\PY@it=\textit\def\PY@tc##1{\textcolor[rgb]{0.25,0.50,0.50}{##1}}}
\@namedef{PY@tok@cm}{\let\PY@it=\textit\def\PY@tc##1{\textcolor[rgb]{0.25,0.50,0.50}{##1}}}
\@namedef{PY@tok@cpf}{\let\PY@it=\textit\def\PY@tc##1{\textcolor[rgb]{0.25,0.50,0.50}{##1}}}
\@namedef{PY@tok@c1}{\let\PY@it=\textit\def\PY@tc##1{\textcolor[rgb]{0.25,0.50,0.50}{##1}}}
\@namedef{PY@tok@cs}{\let\PY@it=\textit\def\PY@tc##1{\textcolor[rgb]{0.25,0.50,0.50}{##1}}}

\def\PYZbs{\char`\\}
\def\PYZus{\char`\_}
\def\PYZob{\char`\{}
\def\PYZcb{\char`\}}
\def\PYZca{\char`\^}
\def\PYZam{\char`\&}
\def\PYZlt{\char`\<}
\def\PYZgt{\char`\>}
\def\PYZsh{\char`\#}
\def\PYZpc{\char`\%}
\def\PYZdl{\char`\$}
\def\PYZhy{\char`\-}
\def\PYZsq{\char`\'}
\def\PYZdq{\char`\"}
\def\PYZti{\char`\~}
% for compatibility with earlier versions
\def\PYZat{@}
\def\PYZlb{[}
\def\PYZrb{]}
\makeatother


    % For linebreaks inside Verbatim environment from package fancyvrb. 
    \makeatletter
        \newbox\Wrappedcontinuationbox 
        \newbox\Wrappedvisiblespacebox 
        \newcommand*\Wrappedvisiblespace {\textcolor{red}{\textvisiblespace}} 
        \newcommand*\Wrappedcontinuationsymbol {\textcolor{red}{\llap{\tiny$\m@th\hookrightarrow$}}} 
        \newcommand*\Wrappedcontinuationindent {3ex } 
        \newcommand*\Wrappedafterbreak {\kern\Wrappedcontinuationindent\copy\Wrappedcontinuationbox} 
        % Take advantage of the already applied Pygments mark-up to insert 
        % potential linebreaks for TeX processing. 
        %        {, <, #, %, $, ' and ": go to next line. 
        %        _, }, ^, &, >, - and ~: stay at end of broken line. 
        % Use of \textquotesingle for straight quote. 
        \newcommand*\Wrappedbreaksatspecials {% 
            \def\PYGZus{\discretionary{\char`\_}{\Wrappedafterbreak}{\char`\_}}% 
            \def\PYGZob{\discretionary{}{\Wrappedafterbreak\char`\{}{\char`\{}}% 
            \def\PYGZcb{\discretionary{\char`\}}{\Wrappedafterbreak}{\char`\}}}% 
            \def\PYGZca{\discretionary{\char`\^}{\Wrappedafterbreak}{\char`\^}}% 
            \def\PYGZam{\discretionary{\char`\&}{\Wrappedafterbreak}{\char`\&}}% 
            \def\PYGZlt{\discretionary{}{\Wrappedafterbreak\char`\<}{\char`\<}}% 
            \def\PYGZgt{\discretionary{\char`\>}{\Wrappedafterbreak}{\char`\>}}% 
            \def\PYGZsh{\discretionary{}{\Wrappedafterbreak\char`\#}{\char`\#}}% 
            \def\PYGZpc{\discretionary{}{\Wrappedafterbreak\char`\%}{\char`\%}}% 
            \def\PYGZdl{\discretionary{}{\Wrappedafterbreak\char`\$}{\char`\$}}% 
            \def\PYGZhy{\discretionary{\char`\-}{\Wrappedafterbreak}{\char`\-}}% 
            \def\PYGZsq{\discretionary{}{\Wrappedafterbreak\textquotesingle}{\textquotesingle}}% 
            \def\PYGZdq{\discretionary{}{\Wrappedafterbreak\char`\"}{\char`\"}}% 
            \def\PYGZti{\discretionary{\char`\~}{\Wrappedafterbreak}{\char`\~}}% 
        } 
        % Some characters . , ; ? ! / are not pygmentized. 
        % This macro makes them "active" and they will insert potential linebreaks 
        \newcommand*\Wrappedbreaksatpunct {% 
            \lccode`\~`\.\lowercase{\def~}{\discretionary{\hbox{\char`\.}}{\Wrappedafterbreak}{\hbox{\char`\.}}}% 
            \lccode`\~`\,\lowercase{\def~}{\discretionary{\hbox{\char`\,}}{\Wrappedafterbreak}{\hbox{\char`\,}}}% 
            \lccode`\~`\;\lowercase{\def~}{\discretionary{\hbox{\char`\;}}{\Wrappedafterbreak}{\hbox{\char`\;}}}% 
            \lccode`\~`\:\lowercase{\def~}{\discretionary{\hbox{\char`\:}}{\Wrappedafterbreak}{\hbox{\char`\:}}}% 
            \lccode`\~`\?\lowercase{\def~}{\discretionary{\hbox{\char`\?}}{\Wrappedafterbreak}{\hbox{\char`\?}}}% 
            \lccode`\~`\!\lowercase{\def~}{\discretionary{\hbox{\char`\!}}{\Wrappedafterbreak}{\hbox{\char`\!}}}% 
            \lccode`\~`\/\lowercase{\def~}{\discretionary{\hbox{\char`\/}}{\Wrappedafterbreak}{\hbox{\char`\/}}}% 
            \catcode`\.\active
            \catcode`\,\active 
            \catcode`\;\active
            \catcode`\:\active
            \catcode`\?\active
            \catcode`\!\active
            \catcode`\/\active 
            \lccode`\~`\~ 	
        }
    \makeatother

    \let\OriginalVerbatim=\Verbatim
    \makeatletter
    \renewcommand{\Verbatim}[1][1]{%
        %\parskip\z@skip
        \sbox\Wrappedcontinuationbox {\Wrappedcontinuationsymbol}%
        \sbox\Wrappedvisiblespacebox {\FV@SetupFont\Wrappedvisiblespace}%
        \def\FancyVerbFormatLine ##1{\hsize\linewidth
            \vtop{\raggedright\hyphenpenalty\z@\exhyphenpenalty\z@
                \doublehyphendemerits\z@\finalhyphendemerits\z@
                \strut ##1\strut}%
        }%
        % If the linebreak is at a space, the latter will be displayed as visible
        % space at end of first line, and a continuation symbol starts next line.
        % Stretch/shrink are however usually zero for typewriter font.
        \def\FV@Space {%
            \nobreak\hskip\z@ plus\fontdimen3\font minus\fontdimen4\font
            \discretionary{\copy\Wrappedvisiblespacebox}{\Wrappedafterbreak}
            {\kern\fontdimen2\font}%
        }%
        
        % Allow breaks at special characters using \PYG... macros.
        \Wrappedbreaksatspecials
        % Breaks at punctuation characters . , ; ? ! and / need catcode=\active 	
        \OriginalVerbatim[#1,codes*=\Wrappedbreaksatpunct]%
    }
    \makeatother

    % Exact colors from NB
    \definecolor{incolor}{HTML}{303F9F}
    \definecolor{outcolor}{HTML}{D84315}
    \definecolor{cellborder}{HTML}{CFCFCF}
    \definecolor{cellbackground}{HTML}{F7F7F7}
    
    % prompt
    \makeatletter
    \newcommand{\boxspacing}{\kern\kvtcb@left@rule\kern\kvtcb@boxsep}
    \makeatother
    \newcommand{\prompt}[4]{
        {\ttfamily\llap{{\color{#2}[#3]:\hspace{3pt}#4}}\vspace{-\baselineskip}}
    }
    

    
    % Prevent overflowing lines due to hard-to-break entities
    \sloppy 
    % Setup hyperref package
    \hypersetup{
      breaklinks=true,  % so long urls are correctly broken across lines
      colorlinks=true,
      urlcolor=urlcolor,
      linkcolor=linkcolor,
      citecolor=citecolor,
      }
    % Slightly bigger margins than the latex defaults
    
    \geometry{verbose,tmargin=1in,bmargin=1in,lmargin=1in,rmargin=1in}
    
    

\begin{document}
    
    \maketitle
    
    

    
    \hypertarget{fuxedsica-numuxe9rica}{%
\section{Física Numérica}\label{fuxedsica-numuxe9rica}}

    \hypertarget{tarea-2}{%
\subsection{Tarea 2}\label{tarea-2}}

    \begin{enumerate}
\def\labelenumi{\arabic{enumi}.}
\item
  \textbf{La fórmula cuádratica}

  Recuerde que la para la ecuación cuadrática: \[ax^2 + bx + c = 0\]
  tenemos las soluciones:
  \[x_{1,2} = \dfrac{-b \pm \sqrt{b^2 - 4ac}}{2a}  \tag{1} \] o también:
  \[x_{1,2}' = \dfrac{2c}{-b \pm \sqrt{b^2 - 4ac}} \tag{2}  \] Una
  rápida inspección muestra que debemos tener cuidado con la cancelación
  sustractiva cuando \(b^2 \gg 4ac\).

  \begin{enumerate}
  \def\labelenumii{(\alph{enumii})}
  \tightlist
  \item
    Escriba un programa que calcule las cuatro soluciones para valores
    arbitrarios de \(a\), \(b\) y \(c\).
  \item
    Investigue cómo los errores en las respuestas calculadas se vuelven
    grandes debido a la cancelación sustractiva (Sugerencia: pruebe con
    \(a = 1\), \(b = 1\) y \(c = 10^{-n},\,\,n = 1,2,3,...\).
  \end{enumerate}
\end{enumerate}

    \begin{itemize}
\tightlist
\item
  \emph{Solución}:
\end{itemize}

\begin{quote}
\begin{enumerate}
\def\labelenumi{(\alph{enumi})}
\tightlist
\item
  Primeramente asumiremos \(a\neq 0\), ya que de forma contraria
  tendríamos una ecuación de la forma \(bx+c=0\) cuya solución es
  trivial. De esta manera, realizaremos dos funciones (una para cada
  tipo de solucón) cuyos parámetros serán los valores de \(a\), \(b\) y
  \(c\), y como resultado nos arrojarán el valor de las raíces
  dependiendo de la fórmula usada. Las funciones son:
\end{enumerate}
\end{quote}

    \begin{tcolorbox}[breakable, size=fbox, boxrule=1pt, pad at break*=1mm,colback=cellbackground, colframe=cellborder]
\prompt{In}{incolor}{50}{\boxspacing}
\begin{Verbatim}[commandchars=\\\{\}]
\PY{k}{def} \PY{n+nf}{raices\PYZus{}1}\PY{p}{(} \PY{n}{a}\PY{p}{:}\PY{n+nb}{float} \PY{p}{,} \PY{n}{b}\PY{p}{:} \PY{n+nb}{float}\PY{p}{,} \PY{n}{c}\PY{p}{:} \PY{n+nb}{float} \PY{p}{)} \PY{o}{\PYZhy{}}\PY{o}{\PYZgt{}} \PY{n+nb}{list}\PY{p}{:}
    \PY{l+s+sd}{\PYZdq{}\PYZdq{}\PYZdq{}Esta función calcula las raíces de la ecuación cuadrática}
\PY{l+s+sd}{    ax\PYZca{}2 + bx + c = 0 usando la fórmula 1 y regresa los valores de las }
\PY{l+s+sd}{    raices\PYZdq{}\PYZdq{}\PYZdq{}}
    
    
    \PY{n}{x1} \PY{o}{=} \PY{p}{(}\PY{o}{\PYZhy{}}\PY{n}{b} \PY{o}{+} \PY{p}{(}\PY{n}{b}\PY{o}{*}\PY{o}{*}\PY{l+m+mi}{2} \PY{o}{\PYZhy{}} \PY{l+m+mi}{4}\PY{o}{*}\PY{n}{a}\PY{o}{*}\PY{n}{c}\PY{p}{)}\PY{o}{*}\PY{o}{*}\PY{p}{(}\PY{l+m+mi}{1}\PY{o}{/}\PY{l+m+mi}{2}\PY{p}{)} \PY{p}{)}\PY{o}{/} \PY{p}{(}\PY{l+m+mi}{2}\PY{o}{*}\PY{n}{a}\PY{p}{)}
    \PY{n}{x2} \PY{o}{=} \PY{p}{(}\PY{o}{\PYZhy{}}\PY{n}{b} \PY{o}{\PYZhy{}} \PY{p}{(}\PY{n}{b}\PY{o}{*}\PY{o}{*}\PY{l+m+mi}{2} \PY{o}{\PYZhy{}} \PY{l+m+mi}{4}\PY{o}{*}\PY{n}{a}\PY{o}{*}\PY{n}{c}\PY{p}{)}\PY{o}{*}\PY{o}{*}\PY{p}{(}\PY{l+m+mi}{1}\PY{o}{/}\PY{l+m+mi}{2}\PY{p}{)} \PY{p}{)}\PY{o}{/} \PY{p}{(}\PY{l+m+mi}{2}\PY{o}{*}\PY{n}{a}\PY{p}{)}
    \PY{k}{return} \PY{n}{x1}\PY{p}{,}\PY{n}{x2}
        
\PY{k}{def} \PY{n+nf}{raices\PYZus{}2}\PY{p}{(} \PY{n}{a} \PY{p}{,} \PY{n}{b}\PY{p}{,} \PY{n}{c} \PY{p}{)} \PY{o}{\PYZhy{}}\PY{o}{\PYZgt{}} \PY{n+nb}{list}\PY{p}{:}
    \PY{l+s+sd}{\PYZdq{}\PYZdq{}\PYZdq{}Esta función calcula las raíces de la ecuación cuadrática}
\PY{l+s+sd}{    ax\PYZca{}2 + bx + c = 0 usando la fórmula 2 y regresa los valores de las raices}
\PY{l+s+sd}{    en forma de lista\PYZdq{}\PYZdq{}\PYZdq{}}
    
    
    \PY{n}{x1} \PY{o}{=} \PY{p}{(}\PY{o}{\PYZhy{}}\PY{l+m+mi}{2} \PY{o}{*} \PY{n}{c}\PY{p}{)}\PY{o}{/}\PY{p}{(} \PY{n}{b} \PY{o}{+} \PY{p}{(}\PY{n}{b}\PY{o}{*}\PY{o}{*}\PY{l+m+mi}{2} \PY{o}{\PYZhy{}} \PY{l+m+mi}{4}\PY{o}{*}\PY{n}{a}\PY{o}{*}\PY{n}{c}\PY{p}{)}\PY{o}{*}\PY{o}{*}\PY{p}{(}\PY{l+m+mi}{1}\PY{o}{/}\PY{l+m+mi}{2}\PY{p}{)} \PY{p}{)}
    \PY{n}{x2} \PY{o}{=} \PY{p}{(}\PY{o}{\PYZhy{}}\PY{l+m+mi}{2} \PY{o}{*} \PY{n}{c}\PY{p}{)}\PY{o}{/}\PY{p}{(} \PY{n}{b} \PY{o}{\PYZhy{}} \PY{p}{(}\PY{n}{b}\PY{o}{*}\PY{o}{*}\PY{l+m+mi}{2} \PY{o}{\PYZhy{}} \PY{l+m+mi}{4}\PY{o}{*}\PY{n}{a}\PY{o}{*}\PY{n}{c}\PY{p}{)}\PY{o}{*}\PY{o}{*}\PY{p}{(}\PY{l+m+mi}{1}\PY{o}{/}\PY{l+m+mi}{2}\PY{p}{)} \PY{p}{)}
    \PY{k}{return} \PY{n}{x1}\PY{p}{,}\PY{n}{x2}
\end{Verbatim}
\end{tcolorbox}

    \begin{quote}
Veamos un ejemplo del uso de las funciones. Tomemos la ecuación:
\[x^2 + x + 2 =0\] Cuyas raíces son
\(x_{1,2}=\dfrac{-1\pm\sqrt{7} i}{2}\). Usando el programa obtenemos:
\end{quote}

    \begin{tcolorbox}[breakable, size=fbox, boxrule=1pt, pad at break*=1mm,colback=cellbackground, colframe=cellborder]
\prompt{In}{incolor}{51}{\boxspacing}
\begin{Verbatim}[commandchars=\\\{\}]
\PY{n}{a} \PY{o}{=} \PY{l+m+mi}{1}
\PY{n}{b} \PY{o}{=} \PY{l+m+mi}{1}
\PY{n}{c} \PY{o}{=} \PY{l+m+mi}{2}
\PY{n}{r1}\PY{p}{,}\PY{n}{r2} \PY{o}{=} \PY{n}{raices\PYZus{}1}\PY{p}{(}\PY{n}{a}\PY{p}{,}\PY{n}{b}\PY{p}{,}\PY{n}{c}\PY{p}{)}
\PY{n}{s1}\PY{p}{,}\PY{n}{s2} \PY{o}{=} \PY{n}{raices\PYZus{}2}\PY{p}{(}\PY{n}{a}\PY{p}{,}\PY{n}{b}\PY{p}{,}\PY{n}{c}\PY{p}{)}

\PY{n+nb}{print}\PY{p}{(}\PY{l+s+sa}{f}\PY{l+s+s2}{\PYZdq{}}\PY{l+s+s2}{Las raíces usando la fórmula (1) son: x\PYZus{}1 = }\PY{l+s+si}{\PYZob{}}\PY{n}{r1}\PY{l+s+si}{:}\PY{l+s+s2}{\PYZlt{}.8}\PY{l+s+si}{\PYZcb{}}\PY{l+s+s2}{ y x\PYZus{}2 = }\PY{l+s+si}{\PYZob{}}\PY{n}{r2}\PY{l+s+si}{:}\PY{l+s+s2}{\PYZlt{}.8}\PY{l+s+si}{\PYZcb{}}\PY{l+s+s2}{\PYZdq{}}\PY{p}{)}
\PY{n+nb}{print}\PY{p}{(}\PY{l+s+sa}{f}\PY{l+s+s2}{\PYZdq{}}\PY{l+s+s2}{Las raíces usando la fórmula (2) son: x}\PY{l+s+s2}{\PYZsq{}}\PY{l+s+s2}{\PYZus{}1 = }\PY{l+s+si}{\PYZob{}}\PY{n}{s1}\PY{l+s+si}{:}\PY{l+s+s2}{\PYZlt{}.8}\PY{l+s+si}{\PYZcb{}}\PY{l+s+s2}{ y x}\PY{l+s+s2}{\PYZsq{}}\PY{l+s+s2}{\PYZus{}2 = }\PY{l+s+si}{\PYZob{}}\PY{n}{s2}\PY{l+s+si}{:}\PY{l+s+s2}{\PYZlt{}.8}\PY{l+s+si}{\PYZcb{}}\PY{l+s+s2}{\PYZdq{}}\PY{p}{)}
\end{Verbatim}
\end{tcolorbox}

    \begin{Verbatim}[commandchars=\\\{\}]
Las raíces usando la fórmula (1) son: x\_1 = (-0.5+1.3228757j) y x\_2 =
(-0.5-1.3228757j)
Las raíces usando la fórmula (2) son: x'\_1 = (-0.5+1.3228757j) y x'\_2 =
(-0.5-1.3228757j)
    \end{Verbatim}

    \emph{Nota: python utiliza \(j\) como el número imaginario.}

    \begin{quote}
Podemos ver que nuestras funciones definidas obtienen una aproximación
adecuada para la ecuación cuadrática que hemos ingresado.
\end{quote}

    \begin{quote}
\begin{enumerate}
\def\labelenumi{(\alph{enumi})}
\setcounter{enumi}{1}
\tightlist
\item
  Para este inciso imprimamos en una tabla las raíces calculadas por las
  fórmulas (1) y (2) para cada valor de \(a\), \(b\) y \(c\), sugerido.
  Además compararemos los resultados para ver como difieren. Para ello
  nos apoyaremos de la libreria \emph{PrettyTable} para que la tabla sea
  visualmente mejor.
\end{enumerate}
\end{quote}

    \begin{tcolorbox}[breakable, size=fbox, boxrule=1pt, pad at break*=1mm,colback=cellbackground, colframe=cellborder]
\prompt{In}{incolor}{65}{\boxspacing}
\begin{Verbatim}[commandchars=\\\{\}]
\PY{k+kn}{from} \PY{n+nn}{prettytable} \PY{k+kn}{import} \PY{n}{PrettyTable}

\PY{n}{tabla} \PY{o}{=} \PY{n}{PrettyTable}\PY{p}{(}\PY{p}{)} \PY{c+c1}{\PYZsh{}Define la tabla}
\PY{n}{tabla}\PY{o}{.}\PY{n}{title} \PY{o}{=} \PY{l+s+s2}{\PYZdq{}}\PY{l+s+s2}{Raíces de la ecuación x\PYZca{}2 + x + 10\PYZca{}(\PYZhy{}n) = 0}\PY{l+s+s2}{\PYZdq{}} \PY{c+c1}{\PYZsh{} Agrega el título a la tabla}
\PY{n}{tabla}\PY{o}{.}\PY{n}{field\PYZus{}names} \PY{o}{=} \PY{p}{[}\PY{l+s+s2}{\PYZdq{}}\PY{l+s+s2}{n}\PY{l+s+s2}{\PYZdq{}}\PY{p}{,}\PY{l+s+s2}{\PYZdq{}}\PY{l+s+s2}{x\PYZus{}1}\PY{l+s+s2}{\PYZdq{}}\PY{p}{,}\PY{l+s+s2}{\PYZdq{}}\PY{l+s+s2}{x}\PY{l+s+s2}{\PYZsq{}}\PY{l+s+s2}{\PYZus{}1}\PY{l+s+s2}{\PYZdq{}}\PY{p}{,}\PY{l+s+s2}{\PYZdq{}}\PY{l+s+s2}{|x\PYZus{}1\PYZhy{}x}\PY{l+s+s2}{\PYZsq{}}\PY{l+s+s2}{\PYZus{}1|}\PY{l+s+s2}{\PYZdq{}}\PY{p}{,}\PY{l+s+s2}{\PYZdq{}}\PY{l+s+s2}{x\PYZus{}2}\PY{l+s+s2}{\PYZdq{}}\PY{p}{,} \PY{l+s+s2}{\PYZdq{}}\PY{l+s+s2}{x}\PY{l+s+s2}{\PYZsq{}}\PY{l+s+s2}{\PYZus{}2}\PY{l+s+s2}{\PYZdq{}}\PY{p}{,}\PY{l+s+s2}{\PYZdq{}}\PY{l+s+s2}{|x\PYZus{}2\PYZhy{}x}\PY{l+s+s2}{\PYZsq{}}\PY{l+s+s2}{\PYZus{}2|}\PY{l+s+s2}{\PYZdq{}}\PY{p}{]}\PY{c+c1}{\PYZsh{} Le damos nombre a los compos }
\PY{n}{n} \PY{o}{=} \PY{l+m+mi}{16} \PY{c+c1}{\PYZsh{}valor del exponente máximo de c = 10\PYZca{}(\PYZhy{}n) }
\PY{n}{a} \PY{o}{=} \PY{l+m+mi}{1}
\PY{n}{b} \PY{o}{=} \PY{l+m+mi}{1}
\PY{k}{for} \PY{n}{i} \PY{o+ow}{in} \PY{n+nb}{range}\PY{p}{(}\PY{l+m+mi}{1}\PY{p}{,}\PY{n}{n}\PY{o}{+}\PY{l+m+mi}{1}\PY{p}{)}\PY{p}{:}
    \PY{n}{c} \PY{o}{=} \PY{l+m+mi}{10}\PY{o}{*}\PY{o}{*}\PY{p}{(}\PY{o}{\PYZhy{}}\PY{n}{i}\PY{p}{)}
    \PY{n}{r1}\PY{p}{,}\PY{n}{r2} \PY{o}{=} \PY{n}{raices\PYZus{}1}\PY{p}{(}\PY{n}{a}\PY{p}{,}\PY{n}{b}\PY{p}{,}\PY{n}{c}\PY{p}{)}
    \PY{n}{s1}\PY{p}{,}\PY{n}{s2} \PY{o}{=} \PY{n}{raices\PYZus{}2}\PY{p}{(}\PY{n}{a}\PY{p}{,}\PY{n}{b}\PY{p}{,}\PY{n}{c}\PY{p}{)}
    
    \PY{n}{tabla}\PY{o}{.}\PY{n}{add\PYZus{}row}\PY{p}{(}\PY{p}{[}\PY{n}{i}\PY{p}{,}
                  \PY{l+s+sa}{f}\PY{l+s+s2}{\PYZdq{}}\PY{l+s+si}{\PYZob{}}\PY{n}{r1}\PY{l+s+si}{:}\PY{l+s+s2}{\PYZlt{}.3e}\PY{l+s+si}{\PYZcb{}}\PY{l+s+s2}{\PYZdq{}}\PY{p}{,}\PY{l+s+sa}{f}\PY{l+s+s2}{\PYZdq{}}\PY{l+s+si}{\PYZob{}}\PY{n}{s1}\PY{l+s+si}{:}\PY{l+s+s2}{\PYZlt{}.3e}\PY{l+s+si}{\PYZcb{}}\PY{l+s+s2}{\PYZdq{}}\PY{p}{,}\PY{l+s+sa}{f}\PY{l+s+s2}{\PYZdq{}}\PY{l+s+si}{\PYZob{}}\PY{n+nb}{abs}\PY{p}{(}\PY{n}{r1}\PY{o}{\PYZhy{}}\PY{n}{s1}\PY{p}{)}\PY{l+s+si}{:}\PY{l+s+s2}{\PYZlt{}.3e}\PY{l+s+si}{\PYZcb{}}\PY{l+s+s2}{\PYZdq{}}\PY{p}{,}
                  \PY{l+s+sa}{f}\PY{l+s+s2}{\PYZdq{}}\PY{l+s+si}{\PYZob{}}\PY{n}{r2}\PY{l+s+si}{:}\PY{l+s+s2}{\PYZlt{}.3e}\PY{l+s+si}{\PYZcb{}}\PY{l+s+s2}{\PYZdq{}}\PY{p}{,}\PY{l+s+sa}{f}\PY{l+s+s2}{\PYZdq{}}\PY{l+s+si}{\PYZob{}}\PY{n}{s2}\PY{l+s+si}{:}\PY{l+s+s2}{\PYZlt{}.3e}\PY{l+s+si}{\PYZcb{}}\PY{l+s+s2}{\PYZdq{}}\PY{p}{,}\PY{l+s+sa}{f}\PY{l+s+s2}{\PYZdq{}}\PY{l+s+si}{\PYZob{}}\PY{n+nb}{abs}\PY{p}{(}\PY{n}{r2}\PY{o}{\PYZhy{}}\PY{n}{s2}\PY{p}{)}\PY{l+s+si}{:}\PY{l+s+s2}{\PYZlt{}.3e}\PY{l+s+si}{\PYZcb{}}\PY{l+s+s2}{\PYZdq{}}\PY{p}{]}\PY{p}{)}\PY{c+c1}{\PYZsh{}Inserta un elemento a la tabla}

\PY{n+nb}{print}\PY{p}{(}\PY{n}{tabla}\PY{p}{)}
\end{Verbatim}
\end{tcolorbox}

    \begin{Verbatim}[commandchars=\\\{\}]
+-------------------------------------------------------------------------------
---+
|                   Raíces de la ecuación x\^{}2 + x + 10\^{}(-n) = 0
|
+----+------------+------------+------------+------------+------------+---------
---+
| n  |    x\_1     |    x'\_1    | |x\_1-x'\_1| |    x\_2     |    x'\_2    |
|x\_2-x'\_2| |
+----+------------+------------+------------+------------+------------+---------
---+
| 1  | -1.127e-01 | -1.127e-01 | 1.388e-17  | -8.873e-01 | -8.873e-01 |
1.110e-16  |
| 2  | -1.010e-02 | -1.010e-02 | 2.082e-17  | -9.899e-01 | -9.899e-01 |
1.998e-15  |
| 3  | -1.001e-03 | -1.001e-03 | 2.429e-17  | -9.990e-01 | -9.990e-01 |
2.420e-14  |
| 4  | -1.000e-04 | -1.000e-04 | 5.557e-18  | -9.999e-01 | -9.999e-01 |
5.562e-14  |
| 5  | -1.000e-05 | -1.000e-05 | 1.663e-17  | -1.000e+00 | -1.000e+00 |
1.662e-12  |
| 6  | -1.000e-06 | -1.000e-06 | 4.634e-18  | -1.000e+00 | -1.000e+00 |
4.634e-12  |
| 7  | -1.000e-07 | -1.000e-07 | 5.119e-18  | -1.000e+00 | -1.000e+00 |
5.119e-11  |
| 8  | -1.000e-08 | -1.000e-08 | 5.759e-18  | -1.000e+00 | -1.000e+00 |
5.759e-10  |
| 9  | -1.000e-09 | -1.000e-09 | 2.623e-17  | -1.000e+00 | -1.000e+00 |
2.623e-08  |
| 10 | -1.000e-10 | -1.000e-10 | 8.264e-18  | -1.000e+00 | -1.000e+00 |
8.264e-08  |
| 11 | -1.000e-11 | -1.000e-11 | 8.273e-19  | -1.000e+00 | -1.000e+00 |
8.273e-08  |
| 12 | -1.000e-12 | -1.000e-12 | 3.339e-17  | -1.000e+00 | -1.000e+00 |
3.339e-05  |
| 13 | -9.998e-14 | -1.000e-13 | 2.442e-17  | -1.000e+00 | -1.000e+00 |
2.442e-04  |
| 14 | -9.992e-15 | -1.000e-14 | 7.993e-18  | -1.000e+00 | -1.001e+00 |
7.999e-04  |
| 15 | -9.992e-16 | -1.000e-15 | 7.993e-19  | -1.000e+00 | -1.001e+00 |
7.999e-04  |
| 16 | -1.110e-16 | -1.000e-16 | 1.102e-17  | -1.000e+00 | -9.007e-01 |
9.928e-02  |
+----+------------+------------+------------+------------+------------+---------
---+
    \end{Verbatim}

    \begin{quote}
Recordemos que para las columnas 4,5,7 y 8 se han usado las ecuaciones
(1) y (2). Llamenos raíces positivas a \(x_1\) y \(x_1'\) ya que para
estas se toma el signo positivo. Mientras que para \(x_2\) y \(x_2'\)
nos referiremos a ellas como raíces negativas.
\end{quote}

\begin{quote}
Primeramente notemos que para \(n<11\) obtenemos valores muy cercanos
para las raíces obtenidas por ambos métodos. Sin embargo, a partir de
\(n=12\) la diferencia empieza a ser notoria. Esto se de que a que para
la raiz \(x_1\) tenemos la fórmula:
\[x_1=\dfrac{-b+\sqrt{b^2-4ac}}{2a}\] Y para los valores que hemos
tomado, a partir de \(n=12\) podemos hacer la suposición de que
\(b^2\gg 4ac\). Por lo que \(\sqrt{b^2-4ac}\approx |b|\). De esa forma
tendríamos: \[x_1\approx\dfrac{-b+|b|}{2a}\] Y como b\textgreater0,
entonces el numerador sería la recta de dos números muy cercanos entre
sí. Por lo que el numerador tendría el problema de que la computadora lo
pueda interpretar como cero. Por otro lado, para \(x_1'\) tenemos la
ecuación \(x_1' = frac{-2c}{b+\sqrt{b^2-4ac}}\) y tomando \(b^2\gg 4ac\)
obtenemos la aproxiamción: \[x_1'\approx\dfrac{-2c}{b+|b|}\] En este
caso no se tendría ese problema, debido a que esta vez se suman los
números y no se restan.
\end{quote}

\begin{quote}
Ahora, si analizamos a \(x_2\) y \(x_2'\) tenemos un caso parecido. Solo
que esta vez las aproximaciones para la suposición de que \(b^2\gg 4ac\)
reultan:
\[x_2\approx\dfrac{-b-|b|}{2a}\quad;\quad x_1'\approx\dfrac{-2c}{b-|b|}\]
En \(x_2\) no hay tanto problema para los valores que consideramos. Sin
embargo, para \(x_2'\) ocurre lo mismo que con \(x_1\). Solo que esta
vez este número cercano a cero está dividiendo, esto ocasionaria que
nuetros resultados comiencen a diverger y nos den raices erróneas o
simplemente el programa marque error al intentar dividir por un número
que la computadora interpreta como cero.
\end{quote}

\begin{quote}
Por estas razones, las diferencias se vuelven cada vez más grandes.
Ocasionando que nuestros errores sean mayores.
\end{quote}

    \begin{enumerate}
\def\labelenumi{\arabic{enumi}.}
\setcounter{enumi}{1}
\item
  \textbf{Funciones de Bessel esféricas}

  \begin{enumerate}
  \def\labelenumii{(\alph{enumii})}
  \tightlist
  \item
    Escriba un programa que utilice las fórmulas de recursión hacia
    arriba (\emph{up}) y hacía abajo (\emph{down}) para calcular
    \(j_l(x)\) para los primeros 25 valores de \(l\) para
    \(x=0.1,1,10\).
    \[j_{l+1}(x) = \dfrac{2l+1}{x} j_l(x) - j_{l-1}(x),\quad up \tag{3}\]
    \[j_{l-1}(x) = \dfrac{2l+1}{x} j_l(x) - j_{l-+}(x),\quad down \tag {4}\]
  \item
    Ajsute su programa para que al menor un método de ``buenos'' valores
    (error relativo de \(10^{-10}\).
  \item
    Compare los metodos con las distintas fórmulas de recursión,
    imprimiendo una tabla con columnas: \(l\), \(j_{l}^{up}\),
    \(j_l^{down}\) y
    \(\dfrac{|j_{l}^{up}-j_l^{down}|}{|j_{l}^{up}| + |j_l^{down}|}\).
  \end{enumerate}
\end{enumerate}

    \begin{itemize}
\tightlist
\item
  \emph{Solución}:
\end{itemize}

\begin{quote}
\begin{enumerate}
\def\labelenumi{(\alph{enumi})}
\tightlist
\item
  Primeramente exportamos las librerías que requerimos:
\end{enumerate}
\end{quote}

    \begin{tcolorbox}[breakable, size=fbox, boxrule=1pt, pad at break*=1mm,colback=cellbackground, colframe=cellborder]
\prompt{In}{incolor}{53}{\boxspacing}
\begin{Verbatim}[commandchars=\\\{\}]
\PY{k+kn}{from} \PY{n+nn}{collections} \PY{k+kn}{import} \PY{n}{deque} 
\PY{k+kn}{from} \PY{n+nn}{numpy} \PY{k+kn}{import} \PY{n}{sin}\PY{p}{,} \PY{n}{cos}
\PY{k+kn}{from} \PY{n+nn}{scipy}\PY{n+nn}{.}\PY{n+nn}{special} \PY{k+kn}{import} \PY{n}{spherical\PYZus{}jn}
\PY{k+kn}{from} \PY{n+nn}{prettytable} \PY{k+kn}{import} \PY{n}{PrettyTable}
\end{Verbatim}
\end{tcolorbox}

    \begin{quote}
De la librería \emph{collections} se importó \emph{deque} que es como
una lista que permite controlar de mejor manera el insertar elementos al
inicio o al final de la misma. De \emph{numpy} las funciones \(\sin\) y
\(\cos\) que se ocuparan para los valores de \(j_0(x)\) y \(j_1(x)\). De
\emph{scipy} se importaron las funciones de Bessel esféricas para
comparar nuestro programa con estas. Finalmete de \emph{prettytabble}
solo la usaremos para hacer nuestras tablas visualmente mejor.
\end{quote}

\begin{quote}
Para resolver esta parte se crearon 2 funciones. La primer función
calcula el valor numérico de \(j_l(x)\) para un \(l\) y \(x\) dados como
parámentros, usando la relación de recurrencia \emph{up}. Mientras que
la segunda función tiene el mismo objetivo, pero en lugar de usar la
relación \emph{up}, utiliza la relación \emph{down} apoyándose del
algoritmo de Miller. Las funciones creadas fueron las siguientes:
\end{quote}

    \begin{tcolorbox}[breakable, size=fbox, boxrule=1pt, pad at break*=1mm,colback=cellbackground, colframe=cellborder]
\prompt{In}{incolor}{54}{\boxspacing}
\begin{Verbatim}[commandchars=\\\{\}]
\PY{k}{def} \PY{n+nf}{bessel\PYZus{}esf\PYZus{}up}\PY{p}{(}\PY{n}{l}\PY{p}{:} \PY{n+nb}{int}\PY{p}{,} \PY{n}{x}\PY{p}{:} \PY{n+nb}{float}\PY{p}{)} \PY{o}{\PYZhy{}}\PY{o}{\PYZgt{}} \PY{n+nb}{float}\PY{p}{:}
    \PY{l+s+sd}{\PYZdq{}\PYZdq{}\PYZdq{}Esta función utiliza una relación de recurrencia para calcular el valor}
\PY{l+s+sd}{    numérico de la función de bessel esférica siguente a partir de las dos }
\PY{l+s+sd}{    anteriores\PYZdq{}\PYZdq{}\PYZdq{}}

    
    \PY{n}{j} \PY{o}{=} \PY{p}{[}\PY{n}{sin}\PY{p}{(}\PY{n}{x}\PY{p}{)}\PY{o}{/}\PY{n}{x}\PY{p}{,} \PY{n}{sin}\PY{p}{(}\PY{n}{x}\PY{p}{)}\PY{o}{/}\PY{n}{x}\PY{o}{*}\PY{o}{*}\PY{l+m+mi}{2} \PY{o}{\PYZhy{}} \PY{n}{cos}\PY{p}{(}\PY{n}{x}\PY{p}{)}\PY{o}{/}\PY{n}{x}\PY{p}{]}
    
    \PY{k}{for} \PY{n}{i} \PY{o+ow}{in} \PY{n+nb}{range} \PY{p}{(}\PY{l+m+mi}{1}\PY{p}{,}\PY{n}{l}\PY{p}{)}\PY{p}{:}
        \PY{n}{j\PYZus{}sig} \PY{o}{=} \PY{p}{(}\PY{l+m+mi}{2}\PY{o}{*}\PY{n}{i}\PY{o}{+}\PY{l+m+mi}{1}\PY{p}{)}\PY{o}{/}\PY{n}{x} \PY{o}{*} \PY{n}{j}\PY{p}{[}\PY{n}{i}\PY{p}{]} \PY{o}{\PYZhy{}} \PY{n}{j}\PY{p}{[}\PY{n}{i}\PY{o}{\PYZhy{}}\PY{l+m+mi}{1}\PY{p}{]}
        \PY{n}{j}\PY{o}{.}\PY{n}{append}\PY{p}{(}\PY{n}{j\PYZus{}sig}\PY{p}{)}
    
    \PY{k}{return} \PY{n}{j}\PY{p}{[}\PY{n}{l}\PY{p}{]}

\PY{k}{def} \PY{n+nf}{bessel\PYZus{}esf\PYZus{}down}\PY{p}{(}\PY{n}{l}\PY{p}{:} \PY{n+nb}{int}\PY{p}{,} \PY{n}{x} \PY{o}{=} \PY{l+m+mi}{1}\PY{p}{)} \PY{o}{\PYZhy{}}\PY{o}{\PYZgt{}} \PY{n+nb}{float}\PY{p}{:}
    \PY{l+s+sd}{\PYZdq{}\PYZdq{}\PYZdq{}Esta función utiliza el algoritmo de Miller para obtener obtener }
\PY{l+s+sd}{    los valores numéricos de las primeros l\PYZlt{}50 funciones de Bessel esféricas\PYZdq{}\PYZdq{}\PYZdq{}}
    
    \PY{n}{n} \PY{o}{=} \PY{l+m+mi}{25}
    \PY{n}{j} \PY{o}{=} \PY{n}{deque}\PY{p}{(}\PY{p}{[}\PY{l+m+mi}{1}\PY{p}{,}\PY{l+m+mi}{2}\PY{p}{]}\PY{p}{)}\PY{c+c1}{\PYZsh{}En la posición 0 siempre tendremos al j\PYZus{}99 y en 1 al j\PYZus{}100}
    
    
    \PY{k}{for} \PY{n}{i} \PY{o+ow}{in} \PY{n+nb}{range}\PY{p}{(}\PY{l+m+mi}{0}\PY{p}{,}\PY{n}{n}\PY{p}{)}\PY{p}{:}
        \PY{n}{j\PYZus{}ant} \PY{o}{=} \PY{p}{(}\PY{l+m+mi}{2}\PY{o}{*}\PY{p}{(}\PY{n}{n} \PY{o}{\PYZhy{}} \PY{n}{i}\PY{p}{)} \PY{o}{+} \PY{l+m+mi}{1}\PY{p}{)}\PY{o}{/}\PY{n}{x} \PY{o}{*} \PY{n}{j}\PY{p}{[}\PY{l+m+mi}{0}\PY{p}{]} \PY{o}{\PYZhy{}} \PY{n}{j}\PY{p}{[}\PY{l+m+mi}{1}\PY{p}{]}
        \PY{n}{j}\PY{o}{.}\PY{n}{appendleft}\PY{p}{(}\PY{n}{j\PYZus{}ant}\PY{p}{)}        
        
    \PY{n}{normalizacion} \PY{o}{=} \PY{p}{(}\PY{n}{sin}\PY{p}{(}\PY{n}{x}\PY{p}{)}\PY{o}{/}\PY{n}{x}\PY{p}{)}\PY{o}{/}\PY{n}{j}\PY{p}{[}\PY{l+m+mi}{0}\PY{p}{]}
    \PY{n}{j}\PY{p}{[}\PY{n}{l}\PY{p}{]} \PY{o}{=} \PY{n}{j}\PY{p}{[}\PY{n}{l}\PY{p}{]}\PY{o}{*}\PY{n}{normalizacion}
    \PY{k}{return} \PY{n}{j}\PY{p}{[}\PY{n}{l}\PY{p}{]}     
\end{Verbatim}
\end{tcolorbox}

    \begin{quote}
En la función \emph{bessel\_esf\_up} lo que se hace es crear una lista,
donde el elemento 0 es \(j_0(x)=\frac{\sin x}{x}\) y el elemento 1 es
\(j_1(x)=\frac{\sin x}{x^2}-\frac{\cos x}{x}\). Luego de ello, con un
ciclo \emph{for} corremos un contador que va de 1 hasta el parámentro
dado \(l\), el cual se usará para calcular la función \(j_l(x)\) usando
la relación \emph{up}. Todos los valores calculdos se irán guardando en
la lista. De forma que al final de iterar el ciclo for, el elemento
\(l\) de la lista será el valor del \(j_l(x)\) buscado.
\end{quote}

\begin{quote}
Por otro lado, en la función \emph{bessel\_esf\_down} se crea una
\emph{deque} que es como una especie de lista. En esta especie de lista
ingresamos los valores de \(j_{25}(x)=1\) y \(j_{26}(x)=2\) en las
posiciones 0 y 1. Estos valores se tomaron arbitrariamente para aplicar
el algoritmo de Miller. Después de ello, con un ciclo \emph{for} que va
de 1 a 24 usamos nuestra relación de recurrencia \emph{down}, la cual se
ha ajustado para los valores que toma la variable \(i\). El
\(j_{l-1}(x)\) obtenido de esta relación se agrega al inicio de la
lista, tomando la posición 0 y recorriendo el resto de los elementos a
la siguiente posición. Por ejemplo, al calcular el \(j_{24}(x)\) e
insertarlo en la lista, se agregará en la posición 0 y el \(j_{25}(x)\)
se recorrerá a la posición 1. Luego, para el cálculo de \(j_{23}(x)\) se
requerirá de \(j_{24}(x)\) y \(j_{25}(x)\) cuyas posiciones serán
j{[}0{]} y j{[}1{]} en la lista. Y así sucesivamente. Esa es la razón
por la que al definir la variable \emph{j\_ant} se usan las posiciones 0
y 1 de la lista.
\end{quote}

\begin{quote}
Una vez que termine de iterar ese ciclo for, en la lista tendremos los
valorer numéricos de usar la relación \emph{down} a partir de los
valores \(j_{25}(x)=1\) y \(j_{26}(x)=2\). Sin embargo, estos valores
aún no nos sirven. Debemos normalizarlos de la siguiente forma:
\[j_l^N (x) = j_l^C\times\dfrac{j_0(x)}{j_0^C(x)}\]
\end{quote}

\begin{quote}
Donde \(j_l^N (x)\) es el valor numérico normalizado (la aproximación
deseada), \(j_l^C\) es el valor numérico que ya contiene la lista en la
posición \(l\), \(j_0^C(x)\) es el valor numérico de la lista en la
posición 0 y \(j_0(x)=\frac{\sin x }{x}\) es la primer función esférica
de Bessel que deberá calcularse para el \(x\) dado. Como queremos que
esta función nos devuelva el valor numérico de la función \(j_l(x)\)
para un \(l\) y \(x\) dados, solo normalizamos el elemento en la
posición \(l\) de la lista y lo regresamos como resultado de nuestra
función.
\end{quote}

    \begin{quote}
Lo sigueinte que haremos, será crear dos funciones cuyo obejtivo será
imprimir una tabla con los valores numéricos de las primeras 25
funciones de Bessel esféricas para un \(x\) dado y a su vez lo comparará
con el valor numérico obtenido por las funciones de Bessel esféricas de
la librería \emph{scipy}. Las funciones son:
\end{quote}

    \begin{tcolorbox}[breakable, size=fbox, boxrule=1pt, pad at break*=1mm,colback=cellbackground, colframe=cellborder]
\prompt{In}{incolor}{60}{\boxspacing}
\begin{Verbatim}[commandchars=\\\{\}]
\PY{k}{def} \PY{n+nf}{tabla\PYZus{}bessel\PYZus{}up}\PY{p}{(}\PY{n}{x}\PY{p}{:} \PY{n+nb}{float}\PY{p}{)}\PY{p}{:}
    \PY{l+s+sd}{\PYZdq{}\PYZdq{}\PYZdq{}Esta función imprime una tabla con los valores numéticos para las}
\PY{l+s+sd}{    primeras 25 funciones de bessel esféricas calculadas por la relación de }
\PY{l+s+sd}{    recurrencia up y también ponr la función de bessel integrada por la }
\PY{l+s+sd}{    librería scipy. Además las compara y obtiene el error relativo.\PYZdq{}\PYZdq{}\PYZdq{}}
    
    \PY{n}{tabla} \PY{o}{=} \PY{n}{PrettyTable}\PY{p}{(}\PY{p}{)} \PY{c+c1}{\PYZsh{} Define la tabla}
    \PY{n}{tabla}\PY{o}{.}\PY{n}{title} \PY{o}{=} \PY{p}{(}\PY{l+s+sa}{f}\PY{l+s+s2}{\PYZdq{}}\PY{l+s+s2}{Bessel esférico método up: x = }\PY{l+s+si}{\PYZob{}}\PY{n}{x}\PY{l+s+si}{\PYZcb{}}\PY{l+s+s2}{\PYZdq{}}\PY{p}{)} \PY{c+c1}{\PYZsh{} Agrega título a la tabla}
    \PY{n}{tabla}\PY{o}{.}\PY{n}{field\PYZus{}names} \PY{o}{=} \PY{p}{[}\PY{l+s+s1}{\PYZsq{}}\PY{l+s+s1}{j\PYZus{}l}\PY{l+s+s1}{\PYZsq{}}\PY{p}{,} \PY{l+s+s1}{\PYZsq{}}\PY{l+s+s1}{Up}\PY{l+s+s1}{\PYZsq{}}\PY{p}{,} \PY{l+s+s1}{\PYZsq{}}\PY{l+s+s1}{Scipy}\PY{l+s+s1}{\PYZsq{}}\PY{p}{,} \PY{l+s+s1}{\PYZsq{}}\PY{l+s+s1}{Error relativo}\PY{l+s+s1}{\PYZsq{}}\PY{p}{]} \PY{c+c1}{\PYZsh{} Le da nombre a las columnas}

    \PY{k}{for} \PY{n}{i} \PY{o+ow}{in} \PY{n+nb}{range}\PY{p}{(}\PY{l+m+mi}{0}\PY{p}{,} \PY{l+m+mi}{26}\PY{p}{)}\PY{p}{:}
        \PY{n}{error} \PY{o}{=} \PY{n+nb}{abs}\PY{p}{(}\PY{n}{bessel\PYZus{}esf\PYZus{}up}\PY{p}{(}\PY{n}{i}\PY{p}{,} \PY{n}{x}\PY{p}{)} \PY{o}{\PYZhy{}} \PY{n}{spherical\PYZus{}jn}\PY{p}{(}\PY{n}{i}\PY{p}{,} \PY{n}{x}\PY{p}{)}\PY{p}{)}\PY{o}{/}\PY{n}{spherical\PYZus{}jn}\PY{p}{(}\PY{n}{i}\PY{p}{,} \PY{n}{x}\PY{p}{)}
        \PY{n}{tabla}\PY{o}{.}\PY{n}{add\PYZus{}row}\PY{p}{(}\PY{p}{[}\PY{l+s+sa}{f}\PY{l+s+s2}{\PYZdq{}}\PY{l+s+s2}{j\PYZus{}}\PY{l+s+si}{\PYZob{}}\PY{n}{i}\PY{l+s+si}{\PYZcb{}}\PY{l+s+s2}{\PYZdq{}}\PY{p}{,}
                    \PY{l+s+sa}{f}\PY{l+s+s2}{\PYZdq{}}\PY{l+s+si}{\PYZob{}}\PY{n}{bessel\PYZus{}esf\PYZus{}up}\PY{p}{(}\PY{n}{i}\PY{p}{,} \PY{n}{x}\PY{p}{)}\PY{l+s+si}{:}\PY{l+s+s2}{\PYZlt{}.8e}\PY{l+s+si}{\PYZcb{}}\PY{l+s+s2}{\PYZdq{}}\PY{p}{,}
                    \PY{l+s+sa}{f}\PY{l+s+s2}{\PYZdq{}}\PY{l+s+si}{\PYZob{}}\PY{n}{spherical\PYZus{}jn}\PY{p}{(}\PY{n}{i}\PY{p}{,} \PY{n}{x}\PY{p}{)}\PY{l+s+si}{:}\PY{l+s+s2}{\PYZlt{}.8e}\PY{l+s+si}{\PYZcb{}}\PY{l+s+s2}{\PYZdq{}}\PY{p}{,} 
                    \PY{l+s+sa}{f}\PY{l+s+s2}{\PYZdq{}}\PY{l+s+si}{\PYZob{}}\PY{n}{error}\PY{l+s+si}{:}\PY{l+s+s2}{\PYZlt{}.8e}\PY{l+s+si}{\PYZcb{}}\PY{l+s+s2}{\PYZdq{}}\PY{p}{]}\PY{p}{)}\PY{c+c1}{\PYZsh{}Insertamos un registro. El número de elementos debe coincidir con el número de columnas   }
    \PY{n+nb}{print}\PY{p}{(}\PY{n}{tabla}\PY{p}{)} \PY{c+c1}{\PYZsh{}Imprime la tabla}
    
\PY{k}{def} \PY{n+nf}{tabla\PYZus{}bessel\PYZus{}down}\PY{p}{(}\PY{n}{x}\PY{p}{:} \PY{n+nb}{float}\PY{p}{)}\PY{p}{:}
    \PY{l+s+sd}{\PYZdq{}\PYZdq{}\PYZdq{}Esta función imprime una tabla con los valores numéricos para las}
\PY{l+s+sd}{    primeras 25 funciones de bessel esféricas calculadas por el algoritmo de }
\PY{l+s+sd}{    Miller.\PYZdq{}\PYZdq{}\PYZdq{}}
    
    \PY{n}{tabla} \PY{o}{=} \PY{n}{PrettyTable}\PY{p}{(}\PY{p}{)} \PY{c+c1}{\PYZsh{} Define la tabla}
    \PY{n}{tabla}\PY{o}{.}\PY{n}{title} \PY{o}{=} \PY{p}{(}\PY{l+s+sa}{f}\PY{l+s+s2}{\PYZdq{}}\PY{l+s+s2}{Bessel esférico método down: x = }\PY{l+s+si}{\PYZob{}}\PY{n}{x}\PY{l+s+si}{\PYZcb{}}\PY{l+s+s2}{\PYZdq{}}\PY{p}{)} \PY{c+c1}{\PYZsh{} Agrega título a la tabla}
    \PY{n}{tabla}\PY{o}{.}\PY{n}{field\PYZus{}names} \PY{o}{=} \PY{p}{[}\PY{l+s+s1}{\PYZsq{}}\PY{l+s+s1}{j\PYZus{}l}\PY{l+s+s1}{\PYZsq{}}\PY{p}{,} \PY{l+s+s1}{\PYZsq{}}\PY{l+s+s1}{Down}\PY{l+s+s1}{\PYZsq{}}\PY{p}{,} \PY{l+s+s1}{\PYZsq{}}\PY{l+s+s1}{Scipy}\PY{l+s+s1}{\PYZsq{}}\PY{p}{,} \PY{l+s+s1}{\PYZsq{}}\PY{l+s+s1}{Error relativo}\PY{l+s+s1}{\PYZsq{}}\PY{p}{]}

    \PY{k}{for} \PY{n}{i} \PY{o+ow}{in} \PY{n+nb}{range}\PY{p}{(}\PY{l+m+mi}{0}\PY{p}{,} \PY{l+m+mi}{26}\PY{p}{)}\PY{p}{:}
        \PY{n}{error} \PY{o}{=} \PY{n+nb}{abs}\PY{p}{(}\PY{n}{bessel\PYZus{}esf\PYZus{}down}\PY{p}{(}\PY{n}{i}\PY{p}{,} \PY{n}{x}\PY{p}{)} \PY{o}{\PYZhy{}} \PY{n}{spherical\PYZus{}jn}\PY{p}{(}\PY{n}{i}\PY{p}{,} \PY{n}{x}\PY{p}{)}\PY{p}{)}\PY{o}{/}\PY{n}{spherical\PYZus{}jn}\PY{p}{(}\PY{n}{i}\PY{p}{,} \PY{n}{x}\PY{p}{)}
        \PY{n}{tabla}\PY{o}{.}\PY{n}{add\PYZus{}row}\PY{p}{(}\PY{p}{[}\PY{l+s+sa}{f}\PY{l+s+s2}{\PYZdq{}}\PY{l+s+s2}{j\PYZus{}}\PY{l+s+si}{\PYZob{}}\PY{n}{i}\PY{l+s+si}{\PYZcb{}}\PY{l+s+s2}{\PYZdq{}}\PY{p}{,}
                    \PY{l+s+sa}{f}\PY{l+s+s2}{\PYZdq{}}\PY{l+s+si}{\PYZob{}}\PY{n}{bessel\PYZus{}esf\PYZus{}down}\PY{p}{(}\PY{n}{i}\PY{p}{,} \PY{n}{x}\PY{p}{)}\PY{l+s+si}{:}\PY{l+s+s2}{\PYZlt{}.8e}\PY{l+s+si}{\PYZcb{}}\PY{l+s+s2}{\PYZdq{}}\PY{p}{,}
                    \PY{l+s+sa}{f}\PY{l+s+s2}{\PYZdq{}}\PY{l+s+si}{\PYZob{}}\PY{n}{spherical\PYZus{}jn}\PY{p}{(}\PY{n}{i}\PY{p}{,} \PY{n}{x}\PY{p}{)}\PY{l+s+si}{:}\PY{l+s+s2}{\PYZlt{}.8e}\PY{l+s+si}{\PYZcb{}}\PY{l+s+s2}{\PYZdq{}}\PY{p}{,} 
                    \PY{l+s+sa}{f}\PY{l+s+s2}{\PYZdq{}}\PY{l+s+si}{\PYZob{}}\PY{n}{error}\PY{l+s+si}{:}\PY{l+s+s2}{\PYZlt{}.8e}\PY{l+s+si}{\PYZcb{}}\PY{l+s+s2}{\PYZdq{}}\PY{p}{]}\PY{p}{)}\PY{c+c1}{\PYZsh{}Insertamos un registro. El número de elementos debe coincidir con el número de columnas   }
    \PY{n+nb}{print}\PY{p}{(}\PY{n}{tabla}\PY{p}{)} \PY{c+c1}{\PYZsh{}Imprime la tabla   }
    
\end{Verbatim}
\end{tcolorbox}

    \begin{quote}
Las funciones anteriores son análogas. En lo único que difieren es en
que una llama a la función \emph{bessel\_esf\_up} para mostrar las
aproximaciones calculadas por la relación \emph{up} y la otra usa la
función \emph{bessel\_esf\_down} para mostrar los valores numéricos
obtenidos por el mecanismo de Miller. Además, ambas funciones utilizan
el elemento \emph{PrettyTable} para crear la tabla y que se vea mejor.
Se ha puesto un comentario a lado de cada comando de la librería
\emph{prettytable} para explicar como editar la tabla e imprimirla.
\end{quote}

\begin{quote}
Veamos el resultado de estas funcines para los valores que se piden: \$
x=0.1,1,10\$.
\end{quote}

    \begin{tcolorbox}[breakable, size=fbox, boxrule=1pt, pad at break*=1mm,colback=cellbackground, colframe=cellborder]
\prompt{In}{incolor}{61}{\boxspacing}
\begin{Verbatim}[commandchars=\\\{\}]
\PY{n}{tabla\PYZus{}bessel\PYZus{}up}\PY{p}{(}\PY{l+m+mf}{0.1}\PY{p}{)}
\PY{n}{tabla\PYZus{}bessel\PYZus{}down}\PY{p}{(}\PY{l+m+mf}{0.1}\PY{p}{)}
\PY{n}{tabla\PYZus{}bessel\PYZus{}up}\PY{p}{(}\PY{l+m+mi}{1}\PY{p}{)}
\PY{n}{tabla\PYZus{}bessel\PYZus{}down}\PY{p}{(}\PY{l+m+mi}{1}\PY{p}{)}
\PY{n}{tabla\PYZus{}bessel\PYZus{}up}\PY{p}{(}\PY{l+m+mi}{10}\PY{p}{)}
\PY{n}{tabla\PYZus{}bessel\PYZus{}down}\PY{p}{(}\PY{l+m+mi}{10}\PY{p}{)}
\end{Verbatim}
\end{tcolorbox}

    \begin{Verbatim}[commandchars=\\\{\}]
+----------------------------------------------------------+
|            Bessel esférico método up: x = 0.1            |
+------+-----------------+----------------+----------------+
| j\_l  |        Up       |     Scipy      | Error relativo |
+------+-----------------+----------------+----------------+
| j\_0  |  9.98334166e-01 | 9.98334166e-01 | 0.00000000e+00 |
| j\_1  |  3.33000119e-02 | 3.33000119e-02 | 4.95932780e-14 |
| j\_2  |  6.66190608e-04 | 6.66190608e-04 | 7.35572052e-11 |
| j\_3  |  9.51851727e-06 | 9.51851972e-06 | 2.57234097e-07 |
| j\_4  |  1.05600670e-07 | 1.05772015e-07 | 1.61994783e-03 |
| j\_5  | -1.44569836e-08 | 9.61631023e-10 | 1.60338157e+01 |
| j\_6  | -1.69586887e-06 | 7.39754109e-12 | 2.29248644e+05 |
| j\_7  | -2.20448496e-04 | 4.93188748e-14 | 4.46986061e+09 |
| j\_8  | -3.30655785e-02 | 2.90120010e-16 | 1.13972071e+14 |
| j\_9  | -5.62092789e+00 | 1.52698569e-18 | 3.68106127e+18 |
| j\_10 | -1.06794323e+03 | 7.27151100e-21 | 1.46866756e+23 |
| j\_11 | -2.24262458e+05 | 3.16158151e-23 | 7.09336318e+27 |
| j\_12 | -5.15792975e+07 | 1.26465134e-25 | 4.07853896e+32 |
| j\_13 | -1.28946001e+10 | 4.68395367e-28 | 2.75293076e+37 |
| j\_14 | -3.48149045e+12 | 1.61517440e-30 | 2.15548887e+42 |
| j\_15 | -1.00961934e+15 | 5.21029094e-33 | 1.93774080e+47 |
| j\_16 | -3.12978513e+17 | 1.57888971e-35 | 1.98226963e+52 |
| j\_17 | -1.03281900e+20 | 4.51114830e-38 | 2.28948136e+57 |
| j\_18 | -3.61483519e+22 | 1.21923772e-40 | 2.96483215e+62 |
| j\_19 | -1.33747869e+25 | 3.12627012e-43 | 4.27819299e+67 |
| j\_20 | -5.21613074e+27 | 7.62509231e-46 | 6.84074439e+72 |
| j\_21 | -2.13860023e+30 | 1.77328645e-48 | 1.20600946e+78 |
| j\_22 | -9.19592883e+32 | 3.94065518e-51 | 2.33360404e+83 |
| j\_23 | -4.13814659e+35 | 8.38440913e-54 | 4.93552560e+88 |
| j\_24 | -1.94491970e+38 | 1.71111075e-56 | 1.13664162e+94 |
| j\_25 | -9.53006515e+40 | 3.35513153e-59 | 2.84044457e+99 |
+------+-----------------+----------------+----------------+
+---------------------------------------------------------+
|           Bessel esférico método down: x = 0.1          |
+------+----------------+----------------+----------------+
| j\_l  |      Down      |     Scipy      | Error relativo |
+------+----------------+----------------+----------------+
| j\_0  | 9.98334166e-01 | 9.98334166e-01 | 0.00000000e+00 |
| j\_1  | 3.33000119e-02 | 3.33000119e-02 | 6.25125353e-16 |
| j\_2  | 6.66190608e-04 | 6.66190608e-04 | 9.76479247e-16 |
| j\_3  | 9.51851972e-06 | 9.51851972e-06 | 1.77975772e-16 |
| j\_4  | 1.05772015e-07 | 1.05772015e-07 | 3.12816433e-15 |
| j\_5  | 9.61631023e-10 | 9.61631023e-10 | 4.30092516e-16 |
| j\_6  | 7.39754109e-12 | 7.39754109e-12 | 4.36790310e-16 |
| j\_7  | 4.93188748e-14 | 4.93188748e-14 | 1.91941339e-15 |
| j\_8  | 2.90120010e-16 | 2.90120010e-16 | 1.52948519e-15 |
| j\_9  | 1.52698569e-18 | 1.52698569e-18 | 2.39639894e-15 |
| j\_10 | 7.27151100e-21 | 7.27151100e-21 | 4.96611866e-15 |
| j\_11 | 3.16158151e-23 | 3.16158151e-23 | 7.43611606e-16 |
| j\_12 | 1.26465134e-25 | 1.26465134e-25 | 4.90166404e-15 |
| j\_13 | 4.68395367e-28 | 4.68395367e-28 | 7.27581464e-15 |
| j\_14 | 1.61517440e-30 | 1.61517440e-30 | 1.95206260e-15 |
| j\_15 | 5.21029094e-33 | 5.21029094e-33 | 8.92992130e-15 |
| j\_16 | 1.57888971e-35 | 1.57888971e-35 | 2.36993791e-15 |
| j\_17 | 4.51114830e-38 | 4.51114830e-38 | 9.25749849e-15 |
| j\_18 | 1.21923772e-40 | 1.21923772e-40 | 3.67946853e-15 |
| j\_19 | 3.12627012e-43 | 3.12627012e-43 | 8.91769022e-15 |
| j\_20 | 7.62509231e-46 | 7.62509231e-46 | 1.04055717e-14 |
| j\_21 | 1.77328645e-48 | 1.77328645e-48 | 1.33655435e-14 |
| j\_22 | 3.94065518e-51 | 3.94065518e-51 | 3.76506543e-15 |
| j\_23 | 8.38440913e-54 | 8.38440913e-54 | 6.67737338e-14 |
| j\_24 | 1.71111078e-56 | 1.71111075e-56 | 1.57396531e-08 |
| j\_25 | 3.36832830e-59 | 3.35513153e-59 | 3.93330950e-03 |
+------+----------------+----------------+----------------+
+----------------------------------------------------------+
|             Bessel esférico método up: x = 1             |
+------+-----------------+----------------+----------------+
| j\_l  |        Up       |     Scipy      | Error relativo |
+------+-----------------+----------------+----------------+
| j\_0  |  8.41470985e-01 | 8.41470985e-01 | 0.00000000e+00 |
| j\_1  |  3.01168679e-01 | 3.01168679e-01 | 5.52957413e-16 |
| j\_2  |  6.20350520e-02 | 6.20350520e-02 | 2.34894253e-15 |
| j\_3  |  9.00658112e-03 | 9.00658112e-03 | 7.56942415e-14 |
| j\_4  |  1.01101581e-03 | 1.01101581e-03 | 4.58188899e-12 |
| j\_5  |  9.25611586e-05 | 9.25611586e-05 | 4.42941880e-10 |
| j\_6  |  7.15693586e-06 | 7.15693631e-06 | 6.23673410e-08 |
| j\_7  |  4.79007658e-07 | 4.79013420e-07 | 1.20281993e-05 |
| j\_8  |  2.81790093e-08 | 2.82649880e-08 | 3.04187902e-03 |
| j\_9  |  3.55007135e-11 | 1.49137650e-09 | 9.76196009e-01 |
| j\_10 | -2.75044958e-08 | 7.11655264e-11 | 3.87486227e+02 |
| j\_11 | -5.77629912e-07 | 3.09955185e-12 | 1.86360170e+05 |
| j\_12 | -1.32579835e-05 | 1.24166260e-13 | 1.06776057e+08 |
| j\_13 | -3.30871957e-04 | 4.60463768e-15 | 7.18562416e+10 |
| j\_14 | -8.92028486e-03 | 1.58957599e-16 | 5.61173856e+13 |
| j\_15 | -2.58357389e-01 | 5.13268612e-18 | 5.03357079e+16 |
| j\_16 | -8.00015878e+00 | 1.55670827e-19 | 5.13915094e+19 |
| j\_17 | -2.63746882e+02 | 4.45117750e-21 | 5.92532834e+22 |
| j\_18 | -9.22314072e+03 | 1.20385574e-22 | 7.66133383e+25 |
| j\_19 | -3.40992460e+05 | 3.08874236e-24 | 1.10398479e+29 |
| j\_20 | -1.32894828e+07 | 7.53779572e-26 | 1.76304629e+32 |
| j\_21 | -5.44527802e+08 | 1.75388258e-27 | 3.10469930e+35 |
| j\_22 | -2.34014060e+10 | 3.89936131e-29 | 6.00134333e+38 |
| j\_23 | -1.05251874e+12 | 8.30011892e-31 | 1.26807670e+42 |
| j\_24 | -4.94449795e+13 | 1.69458017e-32 | 2.91783064e+45 |
| j\_25 | -2.42175148e+15 | 3.32393637e-34 | 7.28579374e+48 |
+------+-----------------+----------------+----------------+
+---------------------------------------------------------+
|            Bessel esférico método down: x = 1           |
+------+----------------+----------------+----------------+
| j\_l  |      Down      |     Scipy      | Error relativo |
+------+----------------+----------------+----------------+
| j\_0  | 8.41470985e-01 | 8.41470985e-01 | 0.00000000e+00 |
| j\_1  | 3.01168679e-01 | 3.01168679e-01 | 3.68638276e-16 |
| j\_2  | 6.20350520e-02 | 6.20350520e-02 | 1.11854406e-16 |
| j\_3  | 9.00658112e-03 | 9.00658112e-03 | 0.00000000e+00 |
| j\_4  | 1.01101581e-03 | 1.01101581e-03 | 1.07238894e-15 |
| j\_5  | 9.25611586e-05 | 9.25611586e-05 | 1.02491900e-15 |
| j\_6  | 7.15693631e-06 | 7.15693631e-06 | 1.65691865e-15 |
| j\_7  | 4.79013420e-07 | 4.79013420e-07 | 1.10517904e-15 |
| j\_8  | 2.82649880e-08 | 2.82649880e-08 | 1.99003380e-15 |
| j\_9  | 1.49137650e-09 | 1.49137650e-09 | 4.15981785e-16 |
| j\_10 | 7.11655264e-11 | 7.11655264e-11 | 3.99552072e-15 |
| j\_11 | 3.09955185e-12 | 3.09955185e-12 | 3.12739495e-15 |
| j\_12 | 1.24166260e-13 | 1.24166260e-13 | 1.62643533e-15 |
| j\_13 | 4.60463768e-15 | 4.60463768e-15 | 5.48220093e-15 |
| j\_14 | 1.58957599e-16 | 1.58957599e-16 | 1.55084774e-15 |
| j\_15 | 5.13268612e-18 | 5.13268612e-18 | 2.70164496e-15 |
| j\_16 | 1.55670827e-19 | 1.55670827e-19 | 4.02083835e-15 |
| j\_17 | 4.45117750e-21 | 4.45117750e-21 | 5.07045417e-15 |
| j\_18 | 1.20385574e-22 | 1.20385574e-22 | 4.10105306e-15 |
| j\_19 | 3.08874236e-24 | 3.08874236e-24 | 3.56787917e-16 |
| j\_20 | 7.53779572e-26 | 7.53779572e-26 | 1.27925025e-14 |
| j\_21 | 1.75388258e-27 | 1.75388258e-27 | 4.90886783e-15 |
| j\_22 | 3.89936131e-29 | 3.89936131e-29 | 3.33764906e-12 |
| j\_23 | 8.30011897e-31 | 8.30011892e-31 | 7.03989579e-09 |
| j\_24 | 1.69460762e-32 | 1.69458017e-32 | 1.61986990e-05 |
| j\_25 | 3.45838291e-34 | 3.32393637e-34 | 4.04479883e-02 |
+------+----------------+----------------+----------------+
+------------------------------------------------------------+
|             Bessel esférico método up: x = 10              |
+------+-----------------+-----------------+-----------------+
| j\_l  |        Up       |      Scipy      |  Error relativo |
+------+-----------------+-----------------+-----------------+
| j\_0  | -5.44021111e-02 | -5.44021111e-02 | -0.00000000e+00 |
| j\_1  |  7.84669418e-02 |  7.84669418e-02 |  0.00000000e+00 |
| j\_2  |  7.79421936e-02 |  7.79421936e-02 |  0.00000000e+00 |
| j\_3  | -3.94958450e-02 | -3.94958450e-02 | -0.00000000e+00 |
| j\_4  | -1.05589285e-01 | -1.05589285e-01 | -0.00000000e+00 |
| j\_5  | -5.55345116e-02 | -5.55345116e-02 | -2.49894838e-16 |
| j\_6  |  4.45013223e-02 |  4.45013223e-02 |  3.11851133e-16 |
| j\_7  |  1.13386231e-01 |  1.13386231e-01 |  0.00000000e+00 |
| j\_8  |  1.25578024e-01 |  1.25578024e-01 |  2.21022555e-16 |
| j\_9  |  1.00096410e-01 |  1.00096410e-01 |  5.54576847e-16 |
| j\_10 |  6.46051545e-02 |  6.46051545e-02 |  3.65175805e-15 |
| j\_11 |  3.55744149e-02 |  3.55744149e-02 |  2.92579397e-15 |
| j\_12 |  1.72159997e-02 |  1.72159997e-02 |  6.04573711e-16 |
| j\_13 |  7.46558448e-03 |  7.46558448e-03 |  1.20828612e-14 |
| j\_14 |  2.94107834e-03 |  2.94107834e-03 |  8.49348951e-14 |
| j\_15 |  1.06354271e-03 |  1.06354271e-03 |  5.97179242e-13 |
| j\_16 |  3.55904074e-04 |  3.55904074e-04 |  4.82904640e-12 |
| j\_17 |  1.10940728e-04 |  1.10940728e-04 |  4.53984692e-11 |
| j\_18 |  3.23884744e-05 |  3.23884744e-05 |  4.91200361e-10 |
| j\_19 |  8.89662722e-06 |  8.89662727e-06 |  6.05033705e-09 |
| j\_20 |  2.30837177e-06 |  2.30837196e-06 |  8.40498801e-08 |
| j\_21 |  5.67697030e-07 |  5.67697772e-07 |  1.30641308e-06 |
| j\_22 |  1.32725463e-07 |  1.32728458e-07 |  2.25653727e-05 |
| j\_23 |  2.95675538e-08 |  2.95802899e-08 |  4.30562185e-04 |
| j\_24 |  6.24203967e-09 |  6.29890453e-09 |  9.02773771e-03 |
| j\_25 |  1.01844059e-09 |  1.28434224e-09 |  2.07033330e-01 |
+------+-----------------+-----------------+-----------------+
+------------------------------------------------------------+
|            Bessel esférico método down: x = 10             |
+------+-----------------+-----------------+-----------------+
| j\_l  |       Down      |      Scipy      |  Error relativo |
+------+-----------------+-----------------+-----------------+
| j\_0  | -5.44021111e-02 | -5.44021111e-02 | -0.00000000e+00 |
| j\_1  |  7.84669418e-02 |  7.84669418e-02 |  1.59175428e-15 |
| j\_2  |  7.79421936e-02 |  7.79421936e-02 |  3.56104625e-16 |
| j\_3  | -3.94958450e-02 | -3.94958450e-02 | -2.98667357e-15 |
| j\_4  | -1.05589285e-01 | -1.05589285e-01 | -1.05145425e-15 |
| j\_5  | -5.55345116e-02 | -5.55345116e-02 | -4.99789677e-16 |
| j\_6  |  4.45013223e-02 |  4.45013223e-02 |  3.27443690e-15 |
| j\_7  |  1.13386231e-01 |  1.13386231e-01 |  1.34633337e-15 |
| j\_8  |  1.25578024e-01 |  1.25578024e-01 |  6.63067664e-16 |
| j\_9  |  1.00096410e-01 |  1.00096410e-01 |  2.77288424e-16 |
| j\_10 |  6.46051545e-02 |  6.46051545e-02 |  4.29618594e-15 |
| j\_11 |  3.55744149e-02 |  3.55744149e-02 |  6.43674673e-15 |
| j\_12 |  1.72159997e-02 |  1.72159997e-02 |  1.47112936e-14 |
| j\_13 |  7.46558448e-03 |  7.46558448e-03 |  5.19330667e-14 |
| j\_14 |  2.94107834e-03 |  2.94107834e-03 |  2.72204542e-13 |
| j\_15 |  1.06354271e-03 |  1.06354271e-03 |  1.81865444e-12 |
| j\_16 |  3.55904074e-04 |  3.55904074e-04 |  1.45980257e-11 |
| j\_17 |  1.10940728e-04 |  1.10940728e-04 |  1.37107577e-10 |
| j\_18 |  3.23884744e-05 |  3.23884744e-05 |  1.48331509e-09 |
| j\_19 |  8.89662743e-06 |  8.89662727e-06 |  1.82704900e-08 |
| j\_20 |  2.30837255e-06 |  2.30837196e-06 |  2.53809213e-07 |
| j\_21 |  5.67700012e-07 |  5.67697772e-07 |  3.94503423e-06 |
| j\_22 |  1.32737503e-07 |  1.32728458e-07 |  6.81416689e-05 |
| j\_23 |  2.96187499e-08 |  2.95802899e-08 |  1.30018796e-03 |
| j\_24 |  6.47062189e-09 |  6.29890453e-09 |  2.72614648e-02 |
| j\_25 |  2.08729738e-09 |  1.28434224e-09 |  6.25187840e-01 |
+------+-----------------+-----------------+-----------------+
    \end{Verbatim}

    \begin{quote}
Como podemos ver, los valores calculados por el método \emph{up} no son
buenos. El error relativo es muy grande. A pesar de que para \(x=10\)
fue pequeño, sigue siendo significativo. Por lo que usar la relación de
recurrencia \emph{up} a partir de \(j_0\) y \(j_1\) no es adecuado
realizarlo numéricamente. Para valores pequeños de \(l\), las
aproximaciones empiezan a diverger y arrojan cosas complétamente
erróneas.
\end{quote}

\begin{quote}
Por otro lado, para las funciones calculadas por el mecanismo de Miller
podemos observar que para valores pequeños de \(l\) tenemos errores
relativos muy pequeños. Este error va aumentando conforme aumenta \(l\),
pero aún así para \(l=25\) tenemos un error no tan grande en comparación
con el otro método. Por lo que, los valores calculados a través de este
mecanismo si son adecuados. Para obtener aproximaciones más certeras y
con ello errores sumamente pequeños, debemos comenzar a usar la relación
de recurrencia \emph{down} a partir de \(j_l(x)\) y \(j_l(x)\) con
\(l>25\) si es que se quieren los primeros 25 valores de \(j_l\). Ya que
en nuestra función \emph{bessel\_esf\_down} comenzamos a partir de
\(l=25\) y eso ocasionó que tuvieramos errores más grandes para las
\(l's\) cercanas a 25. Así que, si se quiere tener errores más pequeños
debemos modificar el valor de \(l\) con el que se comienza a utilizar la
relación *down. Esto se realizará en el siguiente inciso.
\end{quote}

    \begin{quote}
\begin{enumerate}
\def\labelenumi{(\alph{enumi})}
\setcounter{enumi}{1}
\tightlist
\item
  Para resolver este inciso, solo ajustaremos a la función
  \emph{bessel\_esf\_down} ya que al usar el mecanismo de Miller
  presenta buanas aproximaciones. Para lograr que nuestros errores
  relativos sean menores a \(10^{-10}\), como ya se dijo anterioremnte,
  debemos comenzar a usar la relación de recurrencia \emph{down} a
  partir de un \(l\) más grande. El problema es encontrar la mímina
  \(l\) que cumpla eso.
\end{enumerate}
\end{quote}

\begin{quote}
La forma en como se resolvió, fue usar un ciclo while dentro de la
función ajustada. Este ciclo while evaluaría el error relativo del valor
numérico de \(j_{25}(x)\) y se repetiría si este error es mayor al
solicitado. La razón de porque se evalua solo para \(l=25\) es que como
solo nos interesan los primeros 25 valores de \(l\), si \(j_{25}\)
cumple, automaticamente todas las funciones anteriores lo harán ya que
entre más pequeña sea \(l\) mayor es la convergencia al valor deseado.
El código ajustado para calcular los valores numéricos de las primeras
25 funciones eféricas de Bessel y el código ajsutado para imprimir estos
valores en forma de tabla son los siguientes:
\end{quote}

    \begin{tcolorbox}[breakable, size=fbox, boxrule=1pt, pad at break*=1mm,colback=cellbackground, colframe=cellborder]
\prompt{In}{incolor}{57}{\boxspacing}
\begin{Verbatim}[commandchars=\\\{\}]
\PY{k}{def} \PY{n+nf}{bessel\PYZus{}esf\PYZus{}down\PYZus{}ajuste}\PY{p}{(}\PY{n}{x}\PY{p}{:} \PY{n+nb}{float}\PY{p}{)} \PY{o}{\PYZhy{}}\PY{o}{\PYZgt{}} \PY{n}{deque}\PY{p}{:}
    \PY{l+s+sd}{\PYZdq{}\PYZdq{}\PYZdq{}Esta función utiliza una el algoritmo de Miller para obtener obtener }
\PY{l+s+sd}{    los valores numéricos de las primeros 25 funciones de Bessel esféricas }
\PY{l+s+sd}{    con un error relativo menor a 10\PYZca{}(\PYZhy{}10)\PYZdq{}\PYZdq{}\PYZdq{}}
    
    \PY{n}{n} \PY{o}{=} \PY{l+m+mi}{25}  \PY{c+c1}{\PYZsh{}Se inicia en 25 porque queremos los primeros 25 valores}
    \PY{n}{error} \PY{o}{=} \PY{l+m+mi}{1}
    \PY{n}{j} \PY{o}{=} \PY{n}{deque}\PY{p}{(}\PY{p}{[}\PY{l+m+mi}{1}\PY{p}{,}\PY{l+m+mi}{2}\PY{p}{]}\PY{p}{)}
    
    \PY{k}{while} \PY{n}{error} \PY{o}{\PYZgt{}} \PY{l+m+mi}{10}\PY{o}{*}\PY{o}{*}\PY{p}{(}\PY{o}{\PYZhy{}}\PY{l+m+mi}{10}\PY{p}{)}\PY{p}{:}
        \PY{k}{for} \PY{n}{i} \PY{o+ow}{in} \PY{n+nb}{range}\PY{p}{(}\PY{l+m+mi}{0}\PY{p}{,}\PY{n}{n}\PY{p}{)}\PY{p}{:}
            \PY{n}{j\PYZus{}ant} \PY{o}{=} \PY{p}{(}\PY{l+m+mi}{2}\PY{o}{*}\PY{p}{(}\PY{n}{n} \PY{o}{\PYZhy{}} \PY{n}{i}\PY{p}{)} \PY{o}{+} \PY{l+m+mi}{1}\PY{p}{)}\PY{o}{/}\PY{n}{x} \PY{o}{*} \PY{n}{j}\PY{p}{[}\PY{l+m+mi}{0}\PY{p}{]} \PY{o}{\PYZhy{}} \PY{n}{j}\PY{p}{[}\PY{l+m+mi}{1}\PY{p}{]}
            \PY{n}{j}\PY{o}{.}\PY{n}{appendleft}\PY{p}{(}\PY{n}{j\PYZus{}ant}\PY{p}{)}   
        
        \PY{n}{normalizacion} \PY{o}{=} \PY{p}{(}\PY{n}{sin}\PY{p}{(}\PY{n}{x}\PY{p}{)}\PY{o}{/}\PY{n}{x}\PY{p}{)}\PY{o}{/}\PY{n}{j}\PY{p}{[}\PY{l+m+mi}{0}\PY{p}{]}
        \PY{k}{for} \PY{n}{i} \PY{o+ow}{in} \PY{n+nb}{range}\PY{p}{(}\PY{l+m+mi}{0}\PY{p}{,}\PY{l+m+mi}{26}\PY{p}{)}\PY{p}{:}
            \PY{n}{j}\PY{p}{[}\PY{n}{i}\PY{p}{]} \PY{o}{=} \PY{n}{j}\PY{p}{[}\PY{n}{i}\PY{p}{]}\PY{o}{*}\PY{n}{normalizacion}
        
        \PY{n}{error} \PY{o}{=} \PY{n+nb}{abs}\PY{p}{(} \PY{p}{(} \PY{n}{j}\PY{p}{[}\PY{l+m+mi}{25}\PY{p}{]} \PY{o}{\PYZhy{}} \PY{n}{spherical\PYZus{}jn}\PY{p}{(}\PY{l+m+mi}{25}\PY{p}{,} \PY{n}{x}\PY{p}{)} \PY{p}{)}\PY{o}{/} \PY{n}{spherical\PYZus{}jn}\PY{p}{(}\PY{l+m+mi}{25}\PY{p}{,} \PY{n}{x}\PY{p}{)} \PY{p}{)}
        \PY{n+nb}{print}\PY{p}{(}\PY{l+s+sa}{f}\PY{l+s+s2}{\PYZdq{}}\PY{l+s+s2}{l = }\PY{l+s+si}{\PYZob{}}\PY{n}{n}\PY{l+s+si}{\PYZcb{}}\PY{l+s+s2}{\PYZdq{}}\PY{p}{)}
        \PY{n}{n} \PY{o}{+}\PY{o}{=} \PY{l+m+mi}{1}    
    \PY{k}{return} \PY{n}{j}

\PY{k}{def} \PY{n+nf}{tabla\PYZus{}bessel\PYZus{}down\PYZus{}ajuste}\PY{p}{(}\PY{n}{x}\PY{p}{:} \PY{n+nb}{float}\PY{p}{)}\PY{p}{:}
    \PY{l+s+sd}{\PYZdq{}\PYZdq{}\PYZdq{}Esta función imprime una tabla con los valores numéricos para las}
\PY{l+s+sd}{    primeras 25 funciones de bessel esféricas calculadas por el algoritmo de }
\PY{l+s+sd}{    Miller con un error relativo menor a 10\PYZca{}(\PYZhy{}10)\PYZdq{}\PYZdq{}\PYZdq{}}
    
    \PY{n}{tabla} \PY{o}{=} \PY{n}{PrettyTable}\PY{p}{(}\PY{p}{)} \PY{c+c1}{\PYZsh{} Define la tabla}
    \PY{n}{tabla}\PY{o}{.}\PY{n}{title} \PY{o}{=} \PY{p}{(}\PY{l+s+sa}{f}\PY{l+s+s2}{\PYZdq{}}\PY{l+s+s2}{Bessel esférico método down con ajuste: x = }\PY{l+s+si}{\PYZob{}}\PY{n}{x}\PY{l+s+si}{\PYZcb{}}\PY{l+s+s2}{\PYZdq{}}\PY{p}{)} \PY{c+c1}{\PYZsh{} Agrega un título}
    \PY{n}{tabla}\PY{o}{.}\PY{n}{field\PYZus{}names} \PY{o}{=} \PY{p}{[}\PY{l+s+s1}{\PYZsq{}}\PY{l+s+s1}{j\PYZus{}l}\PY{l+s+s1}{\PYZsq{}}\PY{p}{,} \PY{l+s+s1}{\PYZsq{}}\PY{l+s+s1}{Down}\PY{l+s+s1}{\PYZsq{}}\PY{p}{,} \PY{l+s+s1}{\PYZsq{}}\PY{l+s+s1}{Scipy}\PY{l+s+s1}{\PYZsq{}}\PY{p}{,} \PY{l+s+s1}{\PYZsq{}}\PY{l+s+s1}{Error relativo}\PY{l+s+s1}{\PYZsq{}}\PY{p}{]} \PY{c+c1}{\PYZsh{} Agrega el nombre de las columnas }
    \PY{n}{j} \PY{o}{=} \PY{n}{bessel\PYZus{}esf\PYZus{}down\PYZus{}ajuste}\PY{p}{(}\PY{n}{x}\PY{p}{)}
    
    \PY{k}{for} \PY{n}{i} \PY{o+ow}{in} \PY{n+nb}{range}\PY{p}{(}\PY{l+m+mi}{0}\PY{p}{,} \PY{l+m+mi}{26}\PY{p}{)}\PY{p}{:}
        \PY{n}{error} \PY{o}{=} \PY{n+nb}{abs}\PY{p}{(} \PY{p}{(}\PY{n}{j}\PY{p}{[}\PY{n}{i}\PY{p}{]} \PY{o}{\PYZhy{}} \PY{n}{spherical\PYZus{}jn}\PY{p}{(}\PY{n}{i}\PY{p}{,} \PY{n}{x}\PY{p}{)}\PY{p}{)} \PY{o}{/} \PY{n}{spherical\PYZus{}jn}\PY{p}{(}\PY{n}{i}\PY{p}{,} \PY{n}{x}\PY{p}{)} \PY{p}{)}
        \PY{n}{tabla}\PY{o}{.}\PY{n}{add\PYZus{}row}\PY{p}{(}\PY{p}{[}\PY{l+s+sa}{f}\PY{l+s+s2}{\PYZdq{}}\PY{l+s+s2}{j\PYZus{}}\PY{l+s+si}{\PYZob{}}\PY{n}{i}\PY{l+s+si}{\PYZcb{}}\PY{l+s+s2}{\PYZdq{}}\PY{p}{,}
                    \PY{l+s+sa}{f}\PY{l+s+s2}{\PYZdq{}}\PY{l+s+si}{\PYZob{}}\PY{n}{j}\PY{p}{[}\PY{n}{i}\PY{p}{]}\PY{l+s+si}{:}\PY{l+s+s2}{\PYZlt{}.10e}\PY{l+s+si}{\PYZcb{}}\PY{l+s+s2}{\PYZdq{}}\PY{p}{,}
                    \PY{l+s+sa}{f}\PY{l+s+s2}{\PYZdq{}}\PY{l+s+si}{\PYZob{}}\PY{n}{spherical\PYZus{}jn}\PY{p}{(}\PY{n}{i}\PY{p}{,} \PY{n}{x}\PY{p}{)}\PY{l+s+si}{:}\PY{l+s+s2}{\PYZlt{}.10e}\PY{l+s+si}{\PYZcb{}}\PY{l+s+s2}{\PYZdq{}}\PY{p}{,} 
                    \PY{l+s+sa}{f}\PY{l+s+s2}{\PYZdq{}}\PY{l+s+si}{\PYZob{}}\PY{n}{error}\PY{l+s+si}{:}\PY{l+s+s2}{\PYZlt{}.10e}\PY{l+s+si}{\PYZcb{}}\PY{l+s+s2}{\PYZdq{}}\PY{p}{]}\PY{p}{)} \PY{c+c1}{\PYZsh{}Inserta un registro en forma de lista. El número de elemntos debe coincidir con el número de columnas}
    \PY{n+nb}{print}\PY{p}{(}\PY{n}{tabla}\PY{p}{)}              
\end{Verbatim}
\end{tcolorbox}

    \begin{quote}
La función \emph{bessel\_esf\_down} lo que hace es crear una
\emph{deque} que es como una lista y les agrega los elementos arbitarios
\(j_l(x)=1\) y \(j_{l+1}(x)=2\) para comenzar a usar el mecanismo de
Miller. Luego, con ayuda de un contador \(n\) iniciado en 25 y con
ayudar de la variable \emph{error} se inicia un cliclo while. Lo primero
es usar un ciclo for para usar la relación de recurrencia \emph{down} y
calcular los valores de \(j_i(x)\) donde \(i=0,1,...,n-1\). Así
calculamos todas las funciones de Bessel esféricas anteriores a partir
de los valores iniciales \(j_l(x)=1\) y \(j_{l+1}(x)=2\) y las agregamos
a la lista. De forma que en la posición 0 esté \(j_0(x)\), en la
posición 1 esté \(j_1(x)\) y así sucesivamente hasta \(j_{25}(x)\) ya
que solo nos interesan estos valores. Luego de ello, se procede a
realizar la normalización de \(j_0(x), j_1(x),...,j_{25}(x)\).
Enseguida, se calcula el error relativo para \(j_{25}(x)\) y se
reescribe este valor en la variable \emph{error}. Finalmente, nuestro
contador \(n\) se aumenta en 1. Al término de este proceso, nuevamente
se evalua la condición *error \textgreater{} 10**(-10)* en el ciclo
while. Si se cumple, entonces no hemos obtenido la aproximación deseada.
Por lo que se repite todo el proceso del while pero ahora para la \(n\)
que ha sido aumentada en 1. Así sucesivamente hasta que se rompa el
while y obtengamos la aproximación deseada. Cuando eso suceda, la
función regresará la \emph{deque} que se usó en la función, cuyos
primeros 26 elementos (del 0 al 25) tendrán nuestras aproximaciones
deseadas.
\end{quote}

\begin{quote}
Por útlimo, la función \emph{tabla\_bessel\_down\_ajuste} usa la
librería \emph{prettytable} para imprimir la tabla. Esta tabla contendrá
los valores obtenidos por la función anterior y así nuestros errores
relativos serán los solicitados. Veamos las tablas de \(x=0.1,1,10\).
\end{quote}

    \begin{tcolorbox}[breakable, size=fbox, boxrule=1pt, pad at break*=1mm,colback=cellbackground, colframe=cellborder]
\prompt{In}{incolor}{58}{\boxspacing}
\begin{Verbatim}[commandchars=\\\{\}]
\PY{n}{tabla\PYZus{}bessel\PYZus{}down\PYZus{}ajuste}\PY{p}{(}\PY{l+m+mf}{0.1}\PY{p}{)}
\PY{n}{tabla\PYZus{}bessel\PYZus{}down\PYZus{}ajuste}\PY{p}{(}\PY{l+m+mi}{1}\PY{p}{)}
\PY{n}{tabla\PYZus{}bessel\PYZus{}down\PYZus{}ajuste}\PY{p}{(}\PY{l+m+mi}{10}\PY{p}{)}
\end{Verbatim}
\end{tcolorbox}

    \begin{Verbatim}[commandchars=\\\{\}]
l = 25
l = 26
l = 27
+---------------------------------------------------------------+
|        Bessel esférico método down con ajuste: x = 0.1        |
+------+------------------+------------------+------------------+
| j\_l  |       Down       |      Scipy       |  Error relativo  |
+------+------------------+------------------+------------------+
| j\_0  | 9.9833416647e-01 | 9.9833416647e-01 | 0.0000000000e+00 |
| j\_1  | 3.3300011903e-02 | 3.3300011903e-02 | 6.2512535349e-16 |
| j\_2  | 6.6619060845e-04 | 6.6619060845e-04 | 1.1392257878e-15 |
| j\_3  | 9.5185197209e-06 | 9.5185197209e-06 | 3.5595154377e-16 |
| j\_4  | 1.0577201502e-07 | 1.0577201502e-07 | 3.0030377615e-15 |
| j\_5  | 9.6163102329e-10 | 9.6163102329e-10 | 4.3009251601e-16 |
| j\_6  | 7.3975410936e-12 | 7.3975410936e-12 | 4.3679031004e-16 |
| j\_7  | 4.9318874757e-14 | 4.9318874757e-14 | 1.9194133908e-15 |
| j\_8  | 2.9012001025e-16 | 2.9012001025e-16 | 1.5294851906e-15 |
| j\_9  | 1.5269856935e-18 | 1.5269856935e-18 | 2.3963989381e-15 |
| j\_10 | 7.2715109967e-21 | 7.2715109967e-21 | 4.9661186614e-15 |
| j\_11 | 3.1615815052e-23 | 3.1615815052e-23 | 3.7180580317e-16 |
| j\_12 | 1.2646513379e-25 | 1.2646513379e-25 | 4.5385778182e-15 |
| j\_13 | 4.6839536653e-28 | 4.6839536653e-28 | 6.7014082214e-15 |
| j\_14 | 1.6151744028e-30 | 1.6151744028e-30 | 1.3013750669e-15 |
| j\_15 | 5.2102909410e-33 | 5.2102909410e-33 | 8.4046318154e-15 |
| j\_16 | 1.5788897129e-35 | 1.5788897129e-35 | 2.7085004743e-15 |
| j\_17 | 4.5111483007e-38 | 4.5111483007e-38 | 8.7946235683e-15 |
| j\_18 | 1.2192377198e-40 | 1.2192377198e-40 | 3.3449713916e-15 |
| j\_19 | 3.1262701152e-43 | 3.1262701152e-43 | 8.5355035003e-15 |
| j\_20 | 7.6250923124e-46 | 7.6250923124e-46 | 9.7934792491e-15 |
| j\_21 | 1.7732864463e-48 | 1.7732864463e-48 | 1.2851484112e-14 |
| j\_22 | 3.9406551793e-51 | 3.9406551793e-51 | 4.2168732861e-15 |
| j\_23 | 8.3844091285e-54 | 8.3844091285e-54 | 1.1059831688e-15 |
| j\_24 | 1.7111107510e-56 | 1.7111107510e-56 | 6.7476583961e-15 |
| j\_25 | 3.3551315322e-59 | 3.3551315322e-59 | 1.3047181832e-14 |
+------+------------------+------------------+------------------+
l = 25
l = 26
l = 27
l = 28
+---------------------------------------------------------------+
|         Bessel esférico método down con ajuste: x = 1         |
+------+------------------+------------------+------------------+
| j\_l  |       Down       |      Scipy       |  Error relativo  |
+------+------------------+------------------+------------------+
| j\_0  | 8.4147098481e-01 | 8.4147098481e-01 | 0.0000000000e+00 |
| j\_1  | 3.0116867894e-01 | 3.0116867894e-01 | 5.5295741337e-16 |
| j\_2  | 6.2035052011e-02 | 6.2035052011e-02 | 2.2370881232e-16 |
| j\_3  | 9.0065811171e-03 | 9.0065811171e-03 | 0.0000000000e+00 |
| j\_4  | 1.0110158084e-03 | 1.0110158084e-03 | 1.0723889414e-15 |
| j\_5  | 9.2561158611e-05 | 9.2561158611e-05 | 8.7850199972e-16 |
| j\_6  | 7.1569363101e-06 | 7.1569363101e-06 | 1.3018646549e-15 |
| j\_7  | 4.7901341987e-07 | 4.7901341987e-07 | 1.1051790411e-15 |
| j\_8  | 2.8264988022e-08 | 2.8264988022e-08 | 1.9900338047e-15 |
| j\_9  | 1.4913765026e-09 | 1.4913765026e-09 | 2.7732118990e-16 |
| j\_10 | 7.1165526400e-11 | 7.1165526400e-11 | 3.9955207240e-15 |
| j\_11 | 3.0995518548e-12 | 3.0995518548e-12 | 2.9970868226e-15 |
| j\_12 | 1.2416625970e-13 | 1.2416625970e-13 | 1.6264353314e-15 |
| j\_13 | 4.6046376777e-15 | 4.6046376777e-15 | 5.4822009318e-15 |
| j\_14 | 1.5895759875e-16 | 1.5895759875e-16 | 1.7059325147e-15 |
| j\_15 | 5.1326861154e-18 | 5.1326861154e-18 | 2.4014621909e-15 |
| j\_16 | 1.5567082706e-19 | 1.5567082706e-19 | 4.1754859823e-15 |
| j\_17 | 4.4511775038e-21 | 4.4511775038e-21 | 4.9014390310e-15 |
| j\_18 | 1.2038557422e-22 | 1.2038557422e-22 | 3.9057648175e-15 |
| j\_19 | 3.0887423635e-24 | 3.0887423635e-24 | 3.5678791695e-16 |
| j\_20 | 7.5377957222e-26 | 7.5377957222e-26 | 1.2640210849e-14 |
| j\_21 | 1.7538825776e-27 | 1.7538825776e-27 | 6.3406209487e-15 |
| j\_22 | 3.8993613099e-29 | 3.8993613099e-29 | 6.0373513326e-15 |
| j\_23 | 8.3001189151e-31 | 8.3001189151e-31 | 1.0551795091e-14 |
| j\_24 | 1.6945801738e-32 | 1.6945801738e-32 | 0.0000000000e+00 |
| j\_25 | 3.3239363664e-34 | 3.3239363664e-34 | 2.4457391015e-13 |
+------+------------------+------------------+------------------+
l = 25
l = 26
l = 27
l = 28
l = 29
l = 30
l = 31
l = 32
+-----------------------------------------------------------------+
|          Bessel esférico método down con ajuste: x = 10         |
+------+-------------------+-------------------+------------------+
| j\_l  |        Down       |       Scipy       |  Error relativo  |
+------+-------------------+-------------------+------------------+
| j\_0  | -5.4402111089e-02 | -5.4402111089e-02 | 0.0000000000e+00 |
| j\_1  |  7.8466941799e-02 |  7.8466941799e-02 | 7.0744634567e-16 |
| j\_2  |  7.7942193629e-02 |  7.7942193629e-02 | 1.7805231238e-16 |
| j\_3  | -3.9495844984e-02 | -3.9495844984e-02 | 1.2298067644e-15 |
| j\_4  | -1.0558928512e-01 | -1.0558928512e-01 | 6.5715890549e-16 |
| j\_5  | -5.5534511621e-02 | -5.5534511621e-02 | 3.7484225761e-16 |
| j\_6  |  4.4501322334e-02 |  4.4501322334e-02 | 6.2370226681e-16 |
| j\_7  |  1.1338623066e-01 |  1.1338623066e-01 | 4.8957577045e-16 |
| j\_8  |  1.2557802365e-01 |  1.2557802365e-01 | 6.6306766444e-16 |
| j\_9  |  1.0009640955e-01 |  1.0009640955e-01 | 9.7050948274e-16 |
| j\_10 |  6.4605154493e-02 |  6.4605154493e-02 | 4.5109952333e-15 |
| j\_11 |  3.5574414886e-02 |  3.5574414886e-02 | 4.4862174206e-15 |
| j\_12 |  1.7215999745e-02 |  1.7215999745e-02 | 4.6350651183e-15 |
| j\_13 |  7.4655844766e-03 |  7.4655844766e-03 | 4.2987102224e-15 |
| j\_14 |  2.9410783418e-03 |  2.9410783418e-03 | 4.4236924549e-15 |
| j\_15 |  1.0635427146e-03 |  1.0635427146e-03 | 4.2815855554e-15 |
| j\_16 |  3.5590407351e-04 |  3.5590407351e-04 | 4.7218154903e-15 |
| j\_17 |  1.1094072797e-04 |  1.1094072797e-04 | 4.6420826810e-15 |
| j\_18 |  3.2388474389e-05 |  3.2388474389e-05 | 4.6028039766e-15 |
| j\_19 |  8.8966272694e-06 |  8.8966272694e-06 | 4.5699994208e-15 |
| j\_20 |  2.3083719613e-06 |  2.3083719613e-06 | 4.9536837995e-15 |
| j\_21 |  5.6769777198e-07 |  5.6769777198e-07 | 4.8491595607e-15 |
| j\_22 |  1.3272845821e-07 |  1.3272845821e-07 | 4.1879893668e-15 |
| j\_23 |  2.9580289943e-08 |  2.9580289943e-08 | 8.9484517064e-15 |
| j\_24 |  6.2989045263e-09 |  6.2989045263e-09 | 2.9626095365e-13 |
| j\_25 |  1.2843422360e-09 |  1.2843422360e-09 | 6.7998798626e-12 |
+------+-------------------+-------------------+------------------+
    \end{Verbatim}

    \begin{quote}
Como podemos ver, todos los erorres relativos son menores a
\(10^{-10}\). Además note que antes de cada tabla se imprimen los
valores que \(l\) va tomando. Estos valores indican a partir de que
\(l\) se comenzó a usar la relación de recurrencia \emph{down} para
obtener esos errores relativos. Dado que se tuvo que comenzar
forzosamente con \(l=25\) para obtener los primeros 25 valores, la
primer iteración se hace con 25, luego con 26 y así hasta llegar al
\(l\) que cumple con la condición. Lo sorprendente es que requieren muy
pocas iteraciones para logar ese error relativo. Por ejemplo, en la
tabla \(x=0.1\) se comienza con \(l=25\). No se cumple la condición y
ahora lo realiza con \(l=26\). Tampoco se cumple, así que ahora lo hace
con \(l=27\). Pero en esta si se cumple. Así que, de la relación de
recurrencia \emph{down} bastó tomar \(j_{27}(x)=1\) y \(j_{28}(x)=2\)
como valores iniciales del mecanismo de Miller. Esto permitió que
\(j_{25}(x)\) al ser normalizado tuviera un erorr relativo menor a
\(10^{-10}\). Por lo que el ciclo while, ¡solo se repitió 3 veces!. Esto
es algo demasiado sorprendente. Yo pensé que tomaría mucho más en lograr
ese error relativo, pero no fue así. Por lo que el mecanismo de Miller
es una herramienta sumamente poderosa para lograr aproximaciones muy
precisas de las funciones esféricas de Bessel.
\end{quote}

    \begin{quote}
\begin{enumerate}
\def\labelenumi{(\alph{enumi})}
\setcounter{enumi}{2}
\tightlist
\item
  Para este inciso se creo una función que imprima la tabla deseada. Se
  apoyó de la función \emph{bessel\_esf\_up} y
  \emph{bessel\_esf\_dow\_ajuste} para imprimir esta nueva tabla. Esta
  función es muy similar a todas las anteriores que también imprimen una
  tabla. Al igual se usa la libreía \emph{prettytable} para crear la
  tabla. La función y el resultado son los siguientes:
\end{enumerate}
\end{quote}

    \begin{tcolorbox}[breakable, size=fbox, boxrule=1pt, pad at break*=1mm,colback=cellbackground, colframe=cellborder]
\prompt{In}{incolor}{59}{\boxspacing}
\begin{Verbatim}[commandchars=\\\{\}]
\PY{k}{def} \PY{n+nf}{tabla\PYZus{}up\PYZus{}vs\PYZus{}down}\PY{p}{(}\PY{n}{x}\PY{p}{:} \PY{n+nb}{float}\PY{p}{)}\PY{p}{:}
    \PY{l+s+sd}{\PYZdq{}\PYZdq{}\PYZdq{}Esta función imprime una tabla que compara los valores numérocos}
\PY{l+s+sd}{    para las primeras 25 funciones de bessel calculadas por el método up }
\PY{l+s+sd}{    y por el método down ajustado\PYZdq{}\PYZdq{}\PYZdq{}}
    
    \PY{n}{tabla} \PY{o}{=} \PY{n}{PrettyTable}\PY{p}{(}\PY{p}{)} \PY{c+c1}{\PYZsh{}Crea la tabla}
    \PY{n}{tabla}\PY{o}{.}\PY{n}{title} \PY{o}{=} \PY{p}{(}\PY{l+s+sa}{f}\PY{l+s+s2}{\PYZdq{}}\PY{l+s+s2}{Bessel up vs down: x = }\PY{l+s+si}{\PYZob{}}\PY{n}{x}\PY{l+s+si}{\PYZcb{}}\PY{l+s+s2}{\PYZdq{}}\PY{p}{)}\PY{c+c1}{\PYZsh{}Agrega título}
    \PY{n}{tabla}\PY{o}{.}\PY{n}{field\PYZus{}names} \PY{o}{=} \PY{p}{[}\PY{l+s+s1}{\PYZsq{}}\PY{l+s+s1}{l}\PY{l+s+s1}{\PYZsq{}}\PY{p}{,} \PY{l+s+s1}{\PYZsq{}}\PY{l+s+s1}{j\PYZus{}l up}\PY{l+s+s1}{\PYZsq{}}\PY{p}{,} \PY{l+s+s1}{\PYZsq{}}\PY{l+s+s1}{j\PYZus{}l down}\PY{l+s+s1}{\PYZsq{}}\PY{p}{,} \PY{l+s+s1}{\PYZsq{}}\PY{l+s+s1}{Error}\PY{l+s+s1}{\PYZsq{}}\PY{p}{]}\PY{c+c1}{\PYZsh{}Agrega el nombre de las columnas}
    \PY{n}{j} \PY{o}{=} \PY{n}{bessel\PYZus{}esf\PYZus{}down\PYZus{}ajuste}\PY{p}{(}\PY{n}{x}\PY{p}{)}
    
    \PY{k}{for} \PY{n}{i} \PY{o+ow}{in} \PY{n+nb}{range}\PY{p}{(}\PY{l+m+mi}{0}\PY{p}{,} \PY{l+m+mi}{26}\PY{p}{)}\PY{p}{:}
        \PY{n}{error} \PY{o}{=} \PY{n+nb}{abs}\PY{p}{(} \PY{n}{bessel\PYZus{}esf\PYZus{}up}\PY{p}{(}\PY{n}{i}\PY{p}{,} \PY{n}{x}\PY{p}{)} \PY{o}{\PYZhy{}} \PY{n}{j}\PY{p}{[}\PY{n}{i}\PY{p}{]} \PY{p}{)} \PY{o}{/} \PY{p}{(} \PY{n+nb}{abs}\PY{p}{(}\PY{n}{bessel\PYZus{}esf\PYZus{}up}\PY{p}{(}\PY{n}{i}\PY{p}{,} \PY{n}{x}\PY{p}{)}\PY{p}{)} \PY{o}{+} \PY{n+nb}{abs}\PY{p}{(}\PY{n}{j}\PY{p}{[}\PY{n}{i}\PY{p}{]}\PY{p}{)} \PY{p}{)}
        \PY{n}{tabla}\PY{o}{.}\PY{n}{add\PYZus{}row}\PY{p}{(}\PY{p}{[}\PY{l+s+sa}{f}\PY{l+s+s2}{\PYZdq{}}\PY{l+s+si}{\PYZob{}}\PY{n}{i}\PY{l+s+si}{\PYZcb{}}\PY{l+s+s2}{\PYZdq{}}\PY{p}{,}
                    \PY{l+s+sa}{f}\PY{l+s+s2}{\PYZdq{}}\PY{l+s+si}{\PYZob{}}\PY{n}{bessel\PYZus{}esf\PYZus{}up}\PY{p}{(}\PY{n}{i}\PY{p}{,}\PY{n}{x}\PY{p}{)}\PY{l+s+si}{:}\PY{l+s+s2}{\PYZlt{}.10e}\PY{l+s+si}{\PYZcb{}}\PY{l+s+s2}{\PYZdq{}}\PY{p}{,}
                    \PY{l+s+sa}{f}\PY{l+s+s2}{\PYZdq{}}\PY{l+s+si}{\PYZob{}}\PY{n}{j}\PY{p}{[}\PY{n}{i}\PY{p}{]}\PY{l+s+si}{:}\PY{l+s+s2}{\PYZlt{}.10e}\PY{l+s+si}{\PYZcb{}}\PY{l+s+s2}{\PYZdq{}}\PY{p}{,} 
                    \PY{l+s+sa}{f}\PY{l+s+s2}{\PYZdq{}}\PY{l+s+si}{\PYZob{}}\PY{n}{error}\PY{l+s+si}{:}\PY{l+s+s2}{\PYZlt{}.10e}\PY{l+s+si}{\PYZcb{}}\PY{l+s+s2}{\PYZdq{}}\PY{p}{]}\PY{p}{)} \PY{c+c1}{\PYZsh{}Inserta un registro en la tabla. El número de elementos debe coincidir con el número de columnas   }
    \PY{n+nb}{print}\PY{p}{(}\PY{n}{tabla}\PY{p}{)}   


\PY{n}{tabla\PYZus{}up\PYZus{}vs\PYZus{}down}\PY{p}{(}\PY{l+m+mf}{0.1}\PY{p}{)}
\PY{n}{tabla\PYZus{}up\PYZus{}vs\PYZus{}down}\PY{p}{(}\PY{l+m+mi}{1}\PY{p}{)}
\PY{n}{tabla\PYZus{}up\PYZus{}vs\PYZus{}down}\PY{p}{(}\PY{l+m+mi}{10}\PY{p}{)}
\end{Verbatim}
\end{tcolorbox}

    \begin{Verbatim}[commandchars=\\\{\}]
l = 25
l = 26
l = 27
+--------------------------------------------------------------+
|                  Bessel up vs down: x = 0.1                  |
+----+-------------------+------------------+------------------+
| l  |       j\_l up      |     j\_l down     |      Error       |
+----+-------------------+------------------+------------------+
| 0  |  9.9833416647e-01 | 9.9833416647e-01 | 0.0000000000e+00 |
| 1  |  3.3300011903e-02 | 3.3300011903e-02 | 2.4484076345e-14 |
| 2  |  6.6619060840e-04 | 6.6619060845e-04 | 3.6778032976e-11 |
| 3  |  9.5185172724e-06 | 9.5185197209e-06 | 1.2861706541e-07 |
| 4  |  1.0560066988e-07 | 1.0577201502e-07 | 8.1063050260e-04 |
| 5  | -1.4456983610e-08 | 9.6163102329e-10 | 1.0000000000e+00 |
| 6  | -1.6958688670e-06 | 7.3975410936e-12 | 1.0000000000e+00 |
| 7  | -2.2044849573e-04 | 4.9318874757e-14 | 1.0000000000e+00 |
| 8  | -3.3065578490e-02 | 2.9012001025e-16 | 1.0000000000e+00 |
| 9  | -5.6209278948e+00 | 1.5269856935e-18 | 1.0000000000e+00 |
| 10 | -1.0679432344e+03 | 7.2715109967e-21 | 1.0000000000e+00 |
| 11 | -2.2426245830e+05 | 3.1615815052e-23 | 1.0000000000e+00 |
| 12 | -5.1579297467e+07 | 1.2646513379e-25 | 1.0000000000e+00 |
| 13 | -1.2894600104e+10 | 4.6839536653e-28 | 1.0000000000e+00 |
| 14 | -3.4814904488e+12 | 1.6151744028e-30 | 1.0000000000e+00 |
| 15 | -1.0096193356e+15 | 5.2102909410e-33 | 1.0000000000e+00 |
| 16 | -3.1297851253e+17 | 1.5788897129e-35 | 1.0000000000e+00 |
| 17 | -1.0328189952e+20 | 4.5111483007e-38 | 1.0000000000e+00 |
| 18 | -3.6148351852e+22 | 1.2192377198e-40 | 1.0000000000e+00 |
| 19 | -1.3374786904e+25 | 3.1262701152e-43 | 1.0000000000e+00 |
| 20 | -5.2161307440e+27 | 7.6250923124e-46 | 1.0000000000e+00 |
| 21 | -2.1386002303e+30 | 1.7732864463e-48 | 1.0000000000e+00 |
| 22 | -9.1959288288e+32 | 3.9406551793e-51 | 1.0000000000e+00 |
| 23 | -4.1381465870e+35 | 8.3844091285e-54 | 1.0000000000e+00 |
| 24 | -1.9449196999e+38 | 1.7111107510e-56 | 1.0000000000e+00 |
| 25 | -9.5300651483e+40 | 3.3551315322e-59 | 1.0000000000e+00 |
+----+-------------------+------------------+------------------+
l = 25
l = 26
l = 27
l = 28
+--------------------------------------------------------------+
|                   Bessel up vs down: x = 1                   |
+----+-------------------+------------------+------------------+
| l  |       j\_l up      |     j\_l down     |      Error       |
+----+-------------------+------------------+------------------+
| 0  |  8.4147098481e-01 | 8.4147098481e-01 | 0.0000000000e+00 |
| 1  |  3.0116867894e-01 | 3.0116867894e-01 | 0.0000000000e+00 |
| 2  |  6.2035052011e-02 | 6.2035052011e-02 | 1.0626168585e-15 |
| 3  |  9.0065811171e-03 | 9.0065811171e-03 | 3.7847120744e-14 |
| 4  |  1.0110158084e-03 | 1.0110158084e-03 | 2.2904083010e-12 |
| 5  |  9.2561158570e-05 | 9.2561158611e-05 | 2.2147050060e-10 |
| 6  |  7.1569358637e-06 | 7.1569363101e-06 | 3.1183670799e-08 |
| 7  |  4.7900765821e-07 | 4.7901341987e-07 | 6.0141358091e-06 |
| 8  |  2.8179009348e-08 | 2.8264988022e-08 | 1.5232562895e-03 |
| 9  |  3.5500713480e-11 | 1.4913765026e-09 | 9.5349892826e-01 |
| 10 | -2.7504495792e-08 | 7.1165526400e-11 | 1.0000000000e+00 |
| 11 | -5.7762991235e-07 | 3.0995518548e-12 | 1.0000000000e+00 |
| 12 | -1.3257983488e-05 | 1.2416625970e-13 | 1.0000000000e+00 |
| 13 | -3.3087195729e-04 | 4.6046376777e-15 | 1.0000000000e+00 |
| 14 | -8.9202848634e-03 | 1.5895759875e-16 | 1.0000000000e+00 |
| 15 | -2.5835738908e-01 | 5.1326861154e-18 | 1.0000000000e+00 |
| 16 | -8.0001587767e+00 | 1.5567082706e-19 | 1.0000000000e+00 |
| 17 | -2.6374688224e+02 | 4.4511775038e-21 | 1.0000000000e+00 |
| 18 | -9.2231407196e+03 | 1.2038557422e-22 | 1.0000000000e+00 |
| 19 | -3.4099245974e+05 | 3.0887423635e-24 | 1.0000000000e+00 |
| 20 | -1.3289482789e+07 | 7.5377957222e-26 | 1.0000000000e+00 |
| 21 | -5.4452780190e+08 | 1.7538825776e-27 | 1.0000000000e+00 |
| 22 | -2.3401405999e+10 | 3.8993613099e-29 | 1.0000000000e+00 |
| 23 | -1.0525187422e+12 | 8.3001189151e-31 | 1.0000000000e+00 |
| 24 | -4.9444979475e+13 | 1.6945801738e-32 | 1.0000000000e+00 |
| 25 | -2.4217514755e+15 | 3.3239363664e-34 | 1.0000000000e+00 |
+----+-------------------+------------------+------------------+
l = 25
l = 26
l = 27
l = 28
l = 29
l = 30
l = 31
l = 32
+---------------------------------------------------------------+
|                   Bessel up vs down: x = 10                   |
+----+-------------------+-------------------+------------------+
| l  |       j\_l up      |      j\_l down     |      Error       |
+----+-------------------+-------------------+------------------+
| 0  | -5.4402111089e-02 | -5.4402111089e-02 | 0.0000000000e+00 |
| 1  |  7.8466941799e-02 |  7.8466941799e-02 | 3.5372317283e-16 |
| 2  |  7.7942193629e-02 |  7.7942193629e-02 | 8.9026156192e-17 |
| 3  | -3.9495844984e-02 | -3.9495844984e-02 | 6.1490338220e-16 |
| 4  | -1.0558928512e-01 | -1.0558928512e-01 | 3.2857945274e-16 |
| 5  | -5.5534511621e-02 | -5.5534511621e-02 | 6.2473709602e-17 |
| 6  |  4.4501322334e-02 |  4.4501322334e-02 | 4.6777670010e-16 |
| 7  |  1.1338623066e-01 |  1.1338623066e-01 | 2.4478788522e-16 |
| 8  |  1.2557802365e-01 |  1.2557802365e-01 | 2.2102255481e-16 |
| 9  |  1.0009640955e-01 |  1.0009640955e-01 | 2.0796631773e-16 |
| 10 |  6.4605154493e-02 |  6.4605154493e-02 | 4.2961859365e-16 |
| 11 |  3.5574414886e-02 |  3.5574414886e-02 | 7.8021172533e-16 |
| 12 |  1.7215999745e-02 |  1.7215999745e-02 | 2.0152457036e-15 |
| 13 |  7.4655844766e-03 |  7.4655844766e-03 | 8.1907856940e-15 |
| 14 |  2.9410783418e-03 |  2.9410783418e-03 | 4.4679293794e-14 |
| 15 |  1.0635427146e-03 |  1.0635427146e-03 | 3.0073041401e-13 |
| 16 |  3.5590407351e-04 |  3.5590407351e-04 | 2.4168841062e-12 |
| 17 |  1.1094072797e-04 |  1.1094072797e-04 | 2.2701555631e-11 |
| 18 |  3.2388474374e-05 |  3.2388474389e-05 | 2.4560248198e-10 |
| 19 |  8.8966272156e-06 |  8.8966272694e-06 | 3.0251708214e-09 |
| 20 |  2.3083717673e-06 |  2.3083719613e-06 | 4.2024944301e-08 |
| 21 |  5.6769703033e-07 |  5.6769777198e-07 | 6.5320696803e-07 |
| 22 |  1.3272546314e-07 |  1.3272845821e-07 | 1.1282813646e-05 |
| 23 |  2.9567553789e-08 |  2.9580289943e-08 | 2.1532744843e-04 |
| 24 |  6.2420396684e-09 |  6.2989045263e-09 | 4.5343362561e-03 |
| 25 |  1.0184405863e-09 |  1.2843422360e-09 | 1.1546970348e-01 |
+----+-------------------+-------------------+------------------+
    \end{Verbatim}

    \begin{quote}
Como ya se vió antes, el método \emph{up} no fue adecuado para obtener
las aproximaciones deseadas, porque había valores en los que
\(j_l^{up}\) arrojaba un valor muy grande que no tenía nada que ver con
la respuesta correcta. Mientras que con el mecanismo de Miller fue
sencillo controlar las aproximaciones y obtener estimaciones muy
cercanas. Estás diferencias se notan claramente en las columnas 2 y 3 de
las tablas anteriores.
\end{quote}

\begin{quote}
Ahora analicemos más a detalle la columna 4. Esta columan contiene el
siguiente valor:
\[\dfrac{|j_{l}^{up}-j_l^{down}|}{|j_{l}^{up}| + |j_l^{down}|} \tag{5}\]
Podemos ver de las tablas que para algunas \(l's\), se cumple que
\(|j_{l}^{up}|\gg|j_l^{down}|\). Cuando esto sucede, en el cálculo de la
fórmula (5), la computadora puede interpretar lo siguiente:
\[|j_{l}^{up}-j_l^{down}|\approx|j_{l}^{up}|\quad;\quad |j_{l}^{up}| + |j_l^{down}|\approx |j_{l}^{up}| \]
Por lo que ese cálculo arrojaría como respuesta 1. En efecto, esto
sucede en las tablas para \(x=0.1\) y \(x=1\). Todo debido a que los
valores calculados por el método up divergen muy rápido y arroja valores
grandes para \(x\) pequeñas como lo son 0.1 y 1.
\end{quote}


    % Add a bibliography block to the postdoc
    
    
    
\end{document}
